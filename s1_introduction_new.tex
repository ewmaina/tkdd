%%Efficient algorithms for supergrap query processing on graph databases Shou Zhang, Xiaofeng Gao, Weili Wu, Jianzhong Li, Hong Gao, 2009.
The power and importance of graph structures is clearly evident in the numerous and varied use of graphs in such applications as: modeling of proteins
\cite{chi_muntz_nijssen2005}, molecular structures of compounds in chemistry\cite{willet_barnard_john1998}\cite{agrafiotis2007}, organization of entities 
in images\cite{petrakis_faloutsos1997} \cite{burge_kropatsch1999}, representation of components in computer aided design (CAD) drawings \cite{cordella2000}, 
topology of sensor networks\cite{li_wan_wang2003},and social and information networks\cite{cai_he_yan2005}. As a result a tremendous amount of structured 
data is being accumulated in large databases. An essential problem in managing the large amount of graphs and graph queries is the efficiency of query 
processing. In this article we address the problem of scalable enumeration of all subgraph isomorphisms existing in a database of graphs.

A subproblem of the subgraph isomorphism enumeration problem is the issue of deciding whether a graph contains another graph, which is called the 
\textit{subgraph isomorphism problem}. A subgraph isomorphism exist between two 
graphs if there exists one-to-one mapping between the smaller graph and a subgraph of the larger graph such that edge the adjacencies are preserved. 
Testing for subgraph isomorphism for an $n$-node graph in an $m$-node larger graph is a combinatorial matching exercise with $m^n$ possibilities, hence it is 
expensive. Therefore for a given query, evaluating a database for subgraph  isomorphism by individually testing each graph  in the database is 
inefficient.  In fact, the subgraph isomorphism problem is known to be NP-complete \cite{cook1971_np}. Enumeration of all subgraph isomorphisms clearly requires 
solving the subgraph isomorphism problem multiple times, one for each instance of isomorphism at the very least. In  many cases, the graph database is also very 
large which makes it necessary to build a framework to facilitate efficient query processing.

There are two basic approaches that past research has taken regarding the problem of finding graph and subgraph isomorphisms. The first approach is based on 
group theoretic concepts and aims to classify the adjacency matrices of graphs into permutation groups. With this, it is possible to prove that there exists 
a moderately exponential bound for the general graph isomorphism problem \cite{babai1981}. Also, by imposing certain restrictions on graphs it is possible to 
derive algorithms that have polynomial bound on complexity. For example  Luks and Hoffman\cite{hoffmann1982} describe a polynomially bound method for the 
isomorphism detection of graphs with bounded valence. These methods, though theoretically interesting, in practice they have a large constant overhead and 
are therefore impractical. They also apply only to graph isomorphism detection, but not the detection and enumeration of subgraph isomorphism which is 
required for a wide range of applications.

The second approach to graph and subgraph isomorphism is the use of heuristics for graph matching. This approach is more practically oriented and aims 
directly at developing practical procedures. Most of the algorithms are based on search tree with backtracking. One of the first publications in this 
field was Corniel and Gotlieb \cite{corneil1970}. A major improvement of the bactracking method was then presented by Ullmann, who introduced  a refinement 
method which reduces the search space of the backtracking procedure remarkably \cite{ullmann1976}.

The methods for finding subgraph isomorphism mentioned so far work only with two graphs at a time. However in many applications there is more than 
one model graph in a graph database that must be matched with one or more input graphs. Consequently it is necessary to compute subgraph isomorphism algorithm for each input graph-database graph pair resulting in a computation time that increases linearly with the size of the database. This dependency becomes prohibitive when the number of graphs in the database is large.  

One of the few methods in literature for searching for all isomorphic subgraphs is based on the RETE algorithm. This aalgorithm was originally conceived for many pattern many target object matching problem in expert systems and has multiple variations have been proposed in that field. Its application in graph theory however was first proposed by Bunke \cite{bunke_glauser_tran1991} and developed further by Messmer\cite{messmer_bunke2000}.

 
Several   to first construct a database of the model graphs to search for and the process the query over the database. In this
an efficient method to determing all possible matches over a large database is useful. Such a method is the Rete Matching Algorithm proposed by Forgy et. al. 
 in the filed of production rules. These algorithm is efficient because it compiles all the patterns into a tree structure. avoids iterating over the dataset by  
 


We can classify a third, more recent and approach aims to reduce the cost of matching a query graph against a very large library of model graphs by representing the common subgraphs efficiently. One example of this approach was proposed by Messmer and Bunke \cite{messmer_bunke2000}. Their work is based on the RETE match algorithm \cite{Forgy1982} initially concieved for fast production matching in expert systems.

\cite{messmer_bunke2000} . In the new approach the tree is replaced by a network whose nodes contain subgraphs of the model graphs and traversing the 
network is equivalent to constructing a graph from the subgraphs. In this method, a powerful tool for obtaining efficient solutions to graph isomorphism 
detection, enumeration  and  problems is divide and conquer, which is expressed in the form of random graph decomposition. Decomposition techniques are 
an instantiation of the divide and conquer paradigm to overcome redundant work for independent partial problems. The main features of the Network based 
approach are as follows:

\begin{enumerate}
\item Offline decomposition of the model graphs into smaller graphs using  random decomposition
\item Decomposed subgraphs of model graph are saved in a network structure as the database. The process is repeated for all the model graphs in the database.
\item Subgraphs that appear in the same or multiple model graphs multiple times are represented in the network structure only once.
\item During graph matching the subgraphs are matched only once to the input graphs, the algorithm is only sublinearly dependent on the number of model graphs.
\end{enumerate}

For the above algorithms, given a database $D$ comprising of a set of graphs $g=g_1 ,g_2 ,\ldots g_n$ and query $Q$ consisting of a set of graphs 
$q=q_1 ,q_2 ,\ldots q_n$, there are, in general, two kinds of graph search problems that we can define. By far the traditional and the most popular 
one is the subgraph query. Given a graph database $D$ and a subgraph query $Q$ with  a query graph $q$, the answer to $Q$ is the set of $\{g \| g\in D$ 
and $q$ is a subgraph of $g\} $. The second search problem is a graph containment search where Given a graph database $D$ and a subgraph query $Q$ with  
a query graph $q$, the answer to $Q$ is the set of $\{g\ | g\in D$ and $q$ is a supergraph of $g\}$. Graph containment search with exclusion logic has 
recently become a  active area of research \cite{chen2007_cindex} \cite{zhang_gao_wu2011}.  
 

\begin{figure}
\centering
%& -shell-escape -enable-write18
\documentclass{standalone}
\usepackage{tikz}
\usetikzlibrary{external}
%\tikzexternalize
\usetikzlibrary{shapes,arrows}
\usepackage{caption}
\usetikzlibrary{matrix}
%\newcommand{mynodename}[#1]{\mathrm{#1}}\label{•} 
%\newcommand{mylabelleft}[#2]{label={[font=\fontsize{#1}{#1}\selectfont]above left:mynodename{#2}}}

\begin{document}
\begin{tikzpicture}[scale=1.2]
%[every node/.style={draw,circle}]
\tikzstyle{every node}=[draw,shape=circle,minimum size=0.7cm];
\tikzstyle{line} = [draw,-latex]
\colorlet{invisible}{white}
\colorlet{visible}{black}
%\tikzstyle{mylabelsfont=[font=\fontsize{7}{7}\selectfont,yshift=-.2cm];
%\tikzstyle{every label}=[text height=10pt];
\begin{scope}[shift={(0,-0.5)}] 
	\begin{scope}[yshift=0cm]% top row as a group
%	\draw[help lines] (-1,-6) grid (10,6);
	  \node (a) at (60:1cm) [label={[font=\fontsize{8}{8}\selectfont,yshift=-.2cm]above left:$1$}] {a};
	  \node (b) at (0:1cm)  [label={[font=\fontsize{8}{8}\selectfont,yshift=-.2cm]above left:$2$}] {b};
	  \node (c) at (0:0cm)  [label={[font=\fontsize{7}{7}\selectfont,yshift=-.2cm]above left:$3$}] {c};
	
	  \foreach \from/\to in {c/b,b/a}
	    \draw (\from) -- (\to);
	\end{scope}
	\begin{scope}[xshift=2cm]%top row middle
	  \node (a) at (60:1cm) [label={[font=\fontsize{7}{7}\selectfont,yshift=-.2cm]above left:$1$}] {a};
	  \node (b) at (0:1cm)  [label={[font=\fontsize{7}{7}\selectfont,yshift=-.2cm]above left:$2$}] {b};
	  \node (c) at (0:0cm)  [label={[font=\fontsize{7}{7}\selectfont,yshift=-.2cm]above left:$3$}] {c};
	
	  \node [draw=none] (query_label) at (0.5,1.5) {Model Graphs};% Model label
	  \foreach \from/\to in {c/b,b/a,c/a}
	    \draw (\from) -- (\to);
	\end{scope}
	\begin{scope}[xshift=4cm]%top row right
	  \node (a_1) at (0,0) [label={[font=\fontsize{7}{7}\selectfont,yshift=-.2cm]above left:$4$}] {a};
	  \node (b) at (0,.9)  [label={[font=\fontsize{7}{7}\selectfont,yshift=-.2cm]above left:$1$}] {b};
	  \node (c) at (1,0)   [label={[font=\fontsize{8}{8}\selectfont,yshift=-.2cm]above left:$3$}] {c};
	  \node (a_2) at (1,.9) [label={[font=\fontsize{7}{7}\selectfont,yshift=-.2cm]above left:$2$}] {a};
	
	  \foreach \from/\to in {a_1/c,c/b,b/a_2}
	    \draw (\from) -- (\to);
	\end{scope}
\end{scope}
\begin{scope}[shift={(0,0)}] % The arrow and bracket as a group
	\begin{scope}[shift={(0,-0.5)}]%just the arrows
		\path[line] (0.5,-0.5) -- (1.5,-1.5);
		\path[line] (2.1,-0.5) -- (2.1,-1.5);
		\path[line] (4.5,-0.5) -- (3.5,-1.5);
	\end{scope}
	\begin{scope}[shift={(0.5,-1.6)}]%annotations at left arrow
		\matrix (m)[matrix of nodes, column  sep=-1mm,color=visible,row  sep=-1mm, anchor=center,draw=none, nodes={rectangle,color=invisible,draw=none,text width = 2cm} ]{
\node [color=visible] {\{1,2,4\}};&\\
\node[color=visible]{\{3,2,4\}};& \\
};
	\end{scope}
	\begin{scope}[shift={(3.0,-1.6)}]%%annotations at middle arrow
	\matrix (m)[matrix of nodes, column  sep=-1mm,color=visible,row  sep=-1mm, anchor=center,draw=none,nodes={rectangle,color=invisible,draw=none,text width = 2cm} ]{
\node [color=visible] {\{1,2,4\}};&\\
\node[color=visible]{\{3,2,4\}};& \\
	};
	\end{scope}
	\begin{scope}[shift={(4.8,-1.6)}]%%annotations at right arrow
	\matrix (m)[matrix of nodes, column  sep=-1mm,color=visible,row  sep=-1mm, anchor=center,draw=none,nodes={rectangle,color=invisible,draw=none,text width = 2cm} ]{
\node [color=visible] {};&\\
\node[color=visible]{\{2,1,4,3\}};& \\
	};
	\end{scope}
\end{scope}

\begin{scope}[shift={(2,-3.5)}]%bottom row graph
  \node (a_3) at (1,0) [label={[font=\fontsize{7}{7}\selectfont,yshift=-.2cm]above left:$3$}] {a};
  \node (b) at (1,.9)  [label={[font=\fontsize{7}{7}\selectfont,yshift=-.2cm]above left:$2$}] {b};
  \node (c) at (0,0)   [label={[font=\fontsize{7}{7}\selectfont,yshift=-.2cm]above left:$4$}] {c};
  \node (a_4) at (0,.9) [label={[font=\fontsize{7}{7}\selectfont,yshift=-.2cm]above left:$1$}] {a};
  
  \node [draw=none] (query_label) at (.5,-.7) {Query Graph};% query graph label
  \foreach \from/\to in {a_3/c,a_3/b,c/b,c/a_4,b/a_4}
    \draw (\from) -- (\to);
    
\end{scope}
\end{tikzpicture}
\end{document}
\caption{Subgraph isomorphism query}
\label{fig:fig11}
\end{figure}


In this paper, we extend the method proposed by Messer et al.\cite{messmer_bunke2000}, so-called \textit{Network Method,(NA)}, to support \textit{subgraph isomorphism} queries.
Fig.\ref{fig:fig11} shows an example of a subgraph isomorphism query. We also reformulate a  \textit{Fast Network Method } to increase the scalability of this method.
Our main contributions are as follows.

\begin{enumerate}
\item The \textit{Network Method} originally only supports induced subgraph isomorphism query. We extend it  cover \textit{subgraph isomorphism} queries.
\item  In the \textit{Network Method}, much of graph decomposition is performed on graphs at random potentially creating many graph fragments. We add a new recombining 
process after each decomposition which is active when more than a pair of  subgraphs result from the decomposition.  Recombining results in exactly  two larger 
graphs for each decomposition step, where possible.  As a consequence, we are able to drastically reduce the potential rapid increase in matchings. 
\item We formulate and implement  a  \textit{Fast Network Method} on the offline preparation of the database. The new algorithm performs recombination 
on  each recursive decomposition in the  preprocessing during database creation. as well as during the actual query processing. 
\item We formulate and implement the \textit{Fast Network Method} for recombination of the decomposed query during  online query processing on the database.
%Random decomposition of graphs results in the generation of multiple graph fragments. The recombining step swaps nodes and  edges until only two graphs result from each decomposition.  
\end{enumerate}

In this work we refer to Messmer's formulation as the \textit{Network Method}.

%%We present experimental results where we compare our proposed algorithms with two well known subgraph isomorphism algorithms: Messmer et. al 's\cite{messmer} Network Method  that efficiently aggregates multiple graphs in a database and VF2\cite{cordella2001_vf2} Algorithm, based on sequential one-on-one graph isomorphism tests. 

We present experimental results where we compare our proposed algorithms with two well known algorithms: Messmer's\cite{messmer_bunke2000} all subgraph isomorphism 
detection Algorithm that efficiently aggregates multiple graphs in a database and the isomorphism detection only algorithm, VF2\cite{cordella2001_vf2}, which is the 
state-of-the-art on sequential one-on-one isomorphism testing. The results show that the proposed improvements result in an order of magnitude increase in scalability 
over the original \textit{Network Method}  for query graphs of up to 500 nodes to a database of 20,000 graphs. We also show a substantial improvement over the VF2 
algorithm despite the increased workload of detecting all subgraph isomorphisms. Our method is particularly suited to larger query graphs or very larger graph 
databases.
