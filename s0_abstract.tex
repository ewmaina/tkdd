Graphs have become increasingly important structures for representing and understanding complex structures.
Deciding whether a graph is a subgraph of another graph is called the subgraph isomorphism problem. 
In this study we are concerned with the retrieval of all subgraph isomorphisms to a query that are present in a graph database.
We define the retrieval of isomorphic subgraphs to satisfy two conditions. 
 First, a decision whether or not a graph isomorphic to a subgraph of a query graph exists in the graph database.
Second, the actual specification of all graphs present in the database isomorphic to query subgraphs . 
However the retrieval of all subgraph isomorphisms is complicated and expensive because the subgraph isomorphism problem is NP-complete.
Recently, there have been several studies on efficiency of retrieval of graph isomorphisms from a graph database.
Notably, Messmer et al. studied an interesting method for retrieval of induced subgraph isomorphisms of a query graph.
They proposed a so-called \textit{Network Method} that stores graphs  in a network structure  by recursively decomposing the individual graphs, and the algorithms to facilitate retrieval using the structure.

In this paper, we extend the Network Method to cover retrieval of non-induced subgraph isomorphisms of a query graph. We also propose a  \textit{ Fast  Network Method} featuring recursive graph decomposing and recombining of the graph fragments. 
We present experimental results where we compare our proposed algorithms with two well known subgraph isomorphism detection algorithms: Messmer's Network Method that efficiently aggregates multiple graphs in a database
 and VF2 Algorithm, based on sequential one-on-one isomorphism tests. 
The results show that the proposed improvements result in an order of magnitude increase in scalability over the original  NA Method  for query graphs of up to 500 nodes to a database of 20,000 graphs.  
Our method is particularly suited to larger query graphs or very larger graph databases.

%Graphs have become increasingly ubiquitous and very important structures for representing and understanding complex structures. In many cases the decomposition of Graphs and their representation has become central problem affecting the tractability of graph search algorithms. Traditional graph search 
