The subgraph isomorphism problem is a known  NP-complete problem\cite{cook1971_np}.
In order to solve this problem efficiently, many algorithms have been proposed such as Ullman\cite{ullmann1976}, Nauty\cite{mckay1981} and VF2\cite{cordella2001_vf2}.

Recently, there have been several studies of graph search on large graph databases. We can classify these studies into two broad categories.

The first one is retrieval of graphs, typically larger than the query graph, that include the query graph from a graph database. 
In this category graph indexing methods have been extensively proposed. 
For example, GraphGrep\cite{shasha_wang_giugno2002_grapgrep} takes the path as the basic indexing unit while gIndex\cite{yan_yu_han2004_gindex} generates an index composed of frequent subgraphs, except for redundant subgraphs. 
FG-Index\cite{cheng2007_fgindex} also utilizes frequent subgraphs and does not need to calculate subgraph isomorphism if query graph is included in frequent subgraphs.
GDIndex\cite{williams_huan_wang2007_gdindex} takes all subgraphs in graph database as the index features in order to facilitate the retrieval.

The second category is the retrieval of model graphs, typically smaller than the query graph and resemble a part of the query graph. 
A good  example of this is approach is the \textit{NA method}\cite{messmer_bunke2000}. More recently, Chen et. al. proposed a containment search algorithm, cIndex\cite{chen2007_cindex}. 
The cIndex indexing model is based on  \textit{contrast subgraphs} that capture differences between feature graphs derived from database model graphs, and graphs. 
The cIndex definition of containment search makes the  \textit{NA method}, in some respect, the more general case of containment search, where, the index features are derived from the database by random decomposition and the query is applied in a decomposed form as multiple graph fragments. 
In this way the objective of the \textit{Network Algorithm} and cIndex are identical, but cIndex performs aggressive optmisation of the model feature set graphs while the 
\textit{Network Algorithm} relies on a form of  exclusion logic at runtime to reduce the problem space.
