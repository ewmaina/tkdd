%%Efficient algorithms for supergrap query processing on graph databases Shou Zhang, Xiaofeng Gao, Weili Wu, Jianzhong Li, Hong Gao, 2009.
The power and importance of graph structures is clearly evident in the numerous and varied use of graphs in such applications as: modeling of proteins
\cite{chi_muntz_nijssen2005}, molecular structures of compounds in chemistry\cite{willet_barnard_john1998}\cite{agrafiotis2007}, organization of entities 
in images\cite{petrakis_faloutsos1997} \cite{burge_kropatsch1999}, representation of components in computer aided design (CAD) drawings \cite{cordella2000}, 
topology of sensor networks\cite{li_wan_wang2003},and social and information networks\cite{cai_he_yan2005}. As a result a tremendous amount of structured 
data is being accumulated in large databases. An essential problem in managing the large amount of graphs and graph queries is the efficiency of query 
processing. In this article we address the problem of scalable enumeration of all subgraph isomorphisms existing in a database of graphs.

A subproblem of the subgraph isomorphism enumeration problem is the decision problem of whether a graph contains another graph, which is called the 
\textit{subgraph isomorphism problem}. A subgraph isomorphism exist between two graphs if there exists one-to-one mapping between the smaller graph 
and a subgraph of the larger graph such that edge the adjacencies are preserved. 
Testing for subgraph isomorphism for an $n$-node graph in an $m$-node larger graph is a combinatorial matching exercise with $m^n$ possibilities, hence it is 
expensive. Therefore for a given query, evaluating a database for subgraph isomorphism by individually testing each graph in the database is 
inefficient. In fact, the subgraph isomorphism problem is known to be NP-complete \cite{cook1971_np}. Enumeration of all subgraph isomorphisms clearly requires 
solving the subgraph isomorphism problem multiple times, one for each instance of isomorphism at the very least. In many cases, the graph database is also very 
large which makes it necessary to build a framework to facilitate efficient query processing.

There are two basic approaches that past research has taken regarding the problem of finding graph and subgraph isomorphisms. The first approach is based on 
group theoretic concepts and aims to classify the adjacency matrices of graphs into permutation groups. It is then possible to prove that there exists 
a moderately exponential bound for the general graph isomorphism problem \cite{babai1981}. Also, by imposing certain restrictions on graphs it is possible to 
derive algorithms that have polynomial bound on complexity. For example Luks and Hoffman\cite{hoffmann1982} describe a polynomially bound method for the 
isomorphism detection of graphs with bounded valence. These methods, though theoretically interesting, in practice they have a large constant overhead and 
are therefore impractical. They also apply only to graph isomorphism detection, but not the detection and enumeration of all subgraph isomorphism which is 
required for a wide range of applications.

The second approach to graph and subgraph isomorphism is the use of heuristics for graph matching. This approach is more practically oriented and aims 
directly at developing practical procedures. Most of the algorithms are based on search tree with backtracking. One of the first publications in this 
field was Corniel and Gotlieb \cite{corneil1970}. A major improvement of the bactracking method was then presented by Ullmann, who introduced  a refinement 
method which reduces the search space of the backtracking procedure remarkably \cite{ullmann1976}.

These methods for finding subgraph isomorphism mentioned so far work only with two graphs at a time. However in many applications there is a large number of model graphs in a graph database that must be matched with one or more input graphs. Consequently it is necessary to compute subgraph isomorphism algorithm for each input graph-database graph pair resulting in a computation time that increases linearly with the size of the database. This dependency becomes prohibitive when the number of graphs in the database is large.  

%In a new approach the tree is replaced by a network whose nodes contain subgraphs of the model graphs and traversing the 
%network is equivalent to constructing a graph from the subgraphs. In this method, a powerful tool for obtaining efficient solutions to graph isomorphism 
%detection, enumeration  and  problems is divide and conquer, which is expressed in the form of random graph decomposition. Decomposition techniques are 
%an instantiation of the divide and conquer paradigm to overcome redundant work for independent partial problems. The main features of the Network based 
%approach are as follows:

One such method for retrieving isomorphic subgraphs from a large collection of graphs is based on the RETE \cite{forgy1982} matching algorithm. This algorithm was originally developed as a solution to the many pattern many target object matching problem in forward chaining rule based systems. It is well researched in the field of Expert Systems. The application of RETE in graph theory was first proposed by Bunke \cite{bunke_glauser_tran1991}. Subsequently a more efficient variation was proposed by Messmer\cite{messmer_bunke2000}. The main features of the Bunke RETE algorithm are: 

\begin{enumerate}
\item Offline decomposition of scene graphs by deleting graph edges
\item Offline compilation of Network prototype graphs
\item online decomposition of new scene graphs by deleting graph edges
\item Objects to be matched are represented only once by a directed acyclic graph (DAG) network (referred to as a Network of prototype Graphs). Multiple instances are represented using links
\item In the DAG network all objects are searched at most once, the search algorithm is sublinearly dependent on the number of network prototype graphs.
\item Given a scene the Network of prototype graphs is edited to add any new prototype graphs present and previous prototypes that are no longer in the scene.
\end{enumerate}
 
\cite{bunke_glauser_tran1991} developed the their RETE inspired graph matching algorithm to allow matching of dynamically changing scene graphs. To do so the graphs are decomposed by deleting edges to a (possibly very) large number of prototype graphs. These graphs are stored in the DAG during offline preprocessed. For the actual search, a scene graph is input and the DAG outputs all the prototype graphs actually present in the scene by dynamically  adding the new edges and deleting old edges not present in the scene.

In contrast, in \cite{messmer_bunke2000} the DAG is build offline by random decomposition of the database (model) graphs, and the subgraphs are stored in the nodes of the DAG. As a result the structure of the DAG network, hence the decomposed subgraphs, is not unique and varies randomly. However especially for very larger database graphs, the DAG is more likely to form a balanced tree resulting in better search efficiency. In \cite{bunke_glauser_tran1991}, during the search the DAG is not altered in any way. The DAG outputs all instances of all the subgraphs in the input graph isomorphic to graphs in the graph database.  

The main features of \cite{bunke_glauser_tran1991} are:

\begin{enumerate}
\item Offline successive decomposition of the model graphs into smaller graphs using random decomposition
\item Offline compilation of directed acyclic graph.
\item Online decomposition of the query graph.
\item Subgraphs that appear in the same or multiple model graphs multiple times are represented in the network structure only once and linked to multiple times.
\item Graph matching the subgraphs are matched only once to the input graphs, the search algorithm is only sublinearly dependent on the number of model graphs.
\item Search is performed by decomposing the query graph, searching for the individual graphs while recombining the subgraphs till a matching database graph is found or the search fails.
\end{enumerate}


It is obvious that Messmer's algorithm retains all the characteristic of Bunke's algorithm  with the additional advantage of dealing with arbitrary query graphs and this illustrates the effective utility of the RETE matching algorithm to graph matching. There are further differences in the graph search strategies which We consider in detail in the following sections.


We now introduce a new term that relates to the present work. For the above RETE based graph algorithms, given a database $D$ comprising of a set of graphs $g=g_{1} ,g_{2} ,\ldots g_{n}$ and query $Q$ consisting of a set of graphs $q=q_{1} ,q_{2} ,\ldots q_{n}$, there are, in general, two kinds of graph search problems that we can define. By far the traditional and the most popular one is the subgraph query. Given a graph database $D$ and a subgraph query $Q$ with  a query graph $q$, the answer to $Q$ is the set of $\{g \| g\in D$ 
and $q$ is a subgraph of $g\} $. The second search problem is a graph containment search where given a graph database $D$ and a subgraph query $Q$ with  
a query graph $q$, the answer to $Q$ is the set of $\{g\ | g\in D$ and $q$ is a supergraph of $g\}$. Graph containment search with exclusion logic has 
recently become an active area of research \cite{chen2007_cindex} \cite{zhang_gao_wu2011}.  
 

\begin{figure}
\centering
%& -shell-escape -enable-write18
\documentclass{standalone}
\usepackage{tikz}
\usetikzlibrary{external}
%\tikzexternalize
\usetikzlibrary{shapes,arrows}
\usepackage{caption}
\usetikzlibrary{matrix}
%\newcommand{mynodename}[#1]{\mathrm{#1}}\label{•} 
%\newcommand{mylabelleft}[#2]{label={[font=\fontsize{#1}{#1}\selectfont]above left:mynodename{#2}}}

\begin{document}
\begin{tikzpicture}[scale=1.2]
%[every node/.style={draw,circle}]
\tikzstyle{every node}=[draw,shape=circle,minimum size=0.7cm];
\tikzstyle{line} = [draw,-latex]
\colorlet{invisible}{white}
\colorlet{visible}{black}
%\tikzstyle{mylabelsfont=[font=\fontsize{7}{7}\selectfont,yshift=-.2cm];
%\tikzstyle{every label}=[text height=10pt];
\begin{scope}[shift={(0,-0.5)}] 
	\begin{scope}[yshift=0cm]% top row as a group
%	\draw[help lines] (-1,-6) grid (10,6);
	  \node (a) at (60:1cm) [label={[font=\fontsize{8}{8}\selectfont,yshift=-.2cm]above left:$1$}] {a};
	  \node (b) at (0:1cm)  [label={[font=\fontsize{8}{8}\selectfont,yshift=-.2cm]above left:$2$}] {b};
	  \node (c) at (0:0cm)  [label={[font=\fontsize{7}{7}\selectfont,yshift=-.2cm]above left:$3$}] {c};
	
	  \foreach \from/\to in {c/b,b/a}
	    \draw (\from) -- (\to);
	\end{scope}
	\begin{scope}[xshift=2cm]%top row middle
	  \node (a) at (60:1cm) [label={[font=\fontsize{7}{7}\selectfont,yshift=-.2cm]above left:$1$}] {a};
	  \node (b) at (0:1cm)  [label={[font=\fontsize{7}{7}\selectfont,yshift=-.2cm]above left:$2$}] {b};
	  \node (c) at (0:0cm)  [label={[font=\fontsize{7}{7}\selectfont,yshift=-.2cm]above left:$3$}] {c};
	
	  \node [draw=none] (query_label) at (0.5,1.5) {Model Graphs};% Model label
	  \foreach \from/\to in {c/b,b/a,c/a}
	    \draw (\from) -- (\to);
	\end{scope}
	\begin{scope}[xshift=4cm]%top row right
	  \node (a_1) at (0,0) [label={[font=\fontsize{7}{7}\selectfont,yshift=-.2cm]above left:$4$}] {a};
	  \node (b) at (0,.9)  [label={[font=\fontsize{7}{7}\selectfont,yshift=-.2cm]above left:$1$}] {b};
	  \node (c) at (1,0)   [label={[font=\fontsize{8}{8}\selectfont,yshift=-.2cm]above left:$3$}] {c};
	  \node (a_2) at (1,.9) [label={[font=\fontsize{7}{7}\selectfont,yshift=-.2cm]above left:$2$}] {a};
	
	  \foreach \from/\to in {a_1/c,c/b,b/a_2}
	    \draw (\from) -- (\to);
	\end{scope}
\end{scope}
\begin{scope}[shift={(0,0)}] % The arrow and bracket as a group
	\begin{scope}[shift={(0,-0.5)}]%just the arrows
		\path[line] (0.5,-0.5) -- (1.5,-1.5);
		\path[line] (2.1,-0.5) -- (2.1,-1.5);
		\path[line] (4.5,-0.5) -- (3.5,-1.5);
	\end{scope}
	\begin{scope}[shift={(0.5,-1.6)}]%annotations at left arrow
		\matrix (m)[matrix of nodes, column  sep=-1mm,color=visible,row  sep=-1mm, anchor=center,draw=none, nodes={rectangle,color=invisible,draw=none,text width = 2cm} ]{
\node [color=visible] {\{1,2,4\}};&\\
\node[color=visible]{\{3,2,4\}};& \\
};
	\end{scope}
	\begin{scope}[shift={(3.0,-1.6)}]%%annotations at middle arrow
	\matrix (m)[matrix of nodes, column  sep=-1mm,color=visible,row  sep=-1mm, anchor=center,draw=none,nodes={rectangle,color=invisible,draw=none,text width = 2cm} ]{
\node [color=visible] {\{1,2,4\}};&\\
\node[color=visible]{\{3,2,4\}};& \\
	};
	\end{scope}
	\begin{scope}[shift={(4.8,-1.6)}]%%annotations at right arrow
	\matrix (m)[matrix of nodes, column  sep=-1mm,color=visible,row  sep=-1mm, anchor=center,draw=none,nodes={rectangle,color=invisible,draw=none,text width = 2cm} ]{
\node [color=visible] {};&\\
\node[color=visible]{\{2,1,4,3\}};& \\
	};
	\end{scope}
\end{scope}

\begin{scope}[shift={(2,-3.5)}]%bottom row graph
  \node (a_3) at (1,0) [label={[font=\fontsize{7}{7}\selectfont,yshift=-.2cm]above left:$3$}] {a};
  \node (b) at (1,.9)  [label={[font=\fontsize{7}{7}\selectfont,yshift=-.2cm]above left:$2$}] {b};
  \node (c) at (0,0)   [label={[font=\fontsize{7}{7}\selectfont,yshift=-.2cm]above left:$4$}] {c};
  \node (a_4) at (0,.9) [label={[font=\fontsize{7}{7}\selectfont,yshift=-.2cm]above left:$1$}] {a};
  
  \node [draw=none] (query_label) at (.5,-.7) {Query Graph};% query graph label
  \foreach \from/\to in {a_3/c,a_3/b,c/b,c/a_4,b/a_4}
    \draw (\from) -- (\to);
    
\end{scope}
\end{tikzpicture}
\end{document}
\caption{Subgraph isomorphism query}
\label{fig:fig11}
\end{figure}


In this paper, we investigate an efficient method to search subgraphs of query graph in a very large graph database. We propose a new RETE based algorithm for retrieving\textit{subgraph isomorphism} queries that is fast and scalable. Fig.\ref{fig:fig11} shows an example of a subgraph isomorphism query. 
Our main contributions are as follows.

\begin{enumerate}
\item In previous work  RETE based graph search only induced subgraph isomorphism query was considered. We propose an algorithm for \textit{subgraph isomorphism} queries.
\item Messmer's Algorithm performs graph decomposition at random potentially creating many graph fragments. Our algorithm performs a recombining 
process after each decomposition. This process is active when more than one pair of  subgraphs results from the decomposition. Recombining will then results in exactly two larger graphs for each decomposition step.  As a consequence, we are able to drastically reduce the unnecessary increase in potential matchings. 
\item We propose a new method for offline preparation of the database that that incoporates  the decompose graphs recombining process. The new algorithm performs recombination on  each recursive decomposition in the  preprocessing during database creation. 
\item We formulate and implement a new graph search strategy for online query processing that is fast and scalable.
%Random decomposition of graphs results in the generation of multiple graph fragments. The recombining step swaps nodes and  edges until only two graphs result from each decomposition.  
\end{enumerate}

In this work we refer to Bunke et al\cite{bunke_glauser_tran1991} as \textit{Bunke RETE} and the work of Messmer et al\cite{messmer_bunke2000} as \textit{Messmer RETE}.

%%We present experimental results where we compare our proposed algorithms with two well known subgraph isomorphism algorithms: Messmer et. al 's\cite{messmer} Network Method  that efficiently aggregates multiple graphs in a database and VF2\cite{cordella2001_vf2} Algorithm, based on sequential one-on-one graph isomorphism tests. 

We present experimental results where we compare our proposed algorithms with two well known algorithms: Messmer RETE \cite{messmer_bunke2000} all subgraph isomorphism 
detection Algorithm that efficiently aggregates multiple graphs in a database and the isomorphism detection only algorithm, VF2\cite{cordella2001_vf2}, which is the 
state-of-the-art on sequential one-on-one isomorphism detection only. The results show that the proposed improvements result in an order of magnitude increase in scalability 
over the original \textit{Network Method}  for query graphs of up to 500 nodes to a database of 20,000 graphs. We also show a substantial improvement over the VF2 
algorithm despite the increased workload of detecting all subgraph isomorphisms. Our method is particularly suited to larger query graphs or very larger graph 
databases.
