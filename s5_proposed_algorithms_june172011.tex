In this section, we detail the algorithms for the proposed extension  of the \textit{Network algorithm} to implement subgraph query processing, as well as other improvements outlined previously in section 4.2. We also describe in detail improved algorithms for processing of induced subgraph query.

%We now describe the algorithms for processing subgraph query and induced subgraph query in detail.

The algorithms solve two problems, database creation and query processing, hence they consist of two parts. The first part concerns the construction of a dynamic acyclic graph $DAG$ to store the model graphs, their decomposed subgraphs and the related link and state information. We construct a $DAG$ consisting of a set of 5-Tuples, each of which  stores information about a graph, its state, its parents, its children and the Edges connecting the children. Creation of the DAG is performed by recursively decomposing the Model graphs and creating a 5-tuple for each resulting graph to store the information. The 5-tuples can be considered as nodes in the DAG as they are connected by a directed edge from parent to child.
The second part of the Algorithms concerns the process of detecting subgraph isomorphisms using $DAG$ constructed in previous part. 
The $DAG$ is the central data structure for this work and is considered  a global variant in all the algorithms. 

\subsection{DAG definition}
First we make a formal definition to help us write the algorithms for the DAG more concisely.


Let $G=\{g_1 ,\ldots, g_n \}$ be a set of model graph. 
\begin{definition}
A \emph{DAG}  $D(G)$  is a finite set of 5-tuple ($g,s,P,C,E$) constructed by decomposition of a set of model graphs $G$
where $g$ is a graph, $s$ is one of $\{unsolved, dead, alive\}$,  $P$ is a set of one or more parent graphs $\{p^{'},p^{''}\ldots \}$ , $C$ is a set of two child graphs $\{c^{'},c^{''}\}$ of graph $g$, and $E$ 
is a set of edges connecting the child graphs $c{'}$ and $c^{''}$. The $DAG$ has to satisfy the following conditions.
%%g are a set of graphs decomposition of $G$, $D(G)$, is a finite set of $4-tuples$ ($g,g^{'} ,g^{''} ,E$), where


\begin{enumerate}[(1)]
%\item $g$ and $c^{'} ,c^{''} $ are graphs with $c^{'}\subset g$ and $c^{''} \subset g$.
\item  $c^{'}\subset g$ and $c^{''} \subset g$ for each child  $c^{'},c^{''}$ of graph $g$
\item  $g \subset p^{'}$, $g \subset p^{''} \ldots$, for each parent in $p$ of graph $g$
%\item $E$ is a set of edges such that $g=c^{'}\cup_{E} c^{''}$
\item $g=c^{'}\cup_{E} c^{''}$
%\item For each $g_{i}$ in $G$ there exists a $5-tuple$ ($g,s,P,C,E$) $\in$ $D(G)$ 
\item For each $g_{i}$ in $G$ there exists a $5-tuple$ ($g,s,P,C,E$) $\in$ $D(G)$ 
\item For each  $5-tuple$ ($g,s,P,C,E$) $\in$ $D(G)$ there exists no other  $5-tuple$ ($g,s,P,C,E$) $\in$ $D(G)$ with $g=g_1$.
\item For each  $5-tuple$ ($g,s,P,C,E$) $\in$ $D(G)$

\begin{enumerate}[a.]
\item if $c^{'},c^{''}$  consists of more than one vertex, then there exists a corresponding  $5-tuple$ 
%\item if $c^{''}$ consists of more than one vertex, then there exists a  $5-tuple$ ($g_2,s_2,P_2,C_2,E_2$) $\in$ $D(G)$ such that $c^{''}=g_2$ 
%\item if $c^{'}$ consists of one vertex then there exists no  $5-tuple$ ($g_3,s_3,P_3,C_3,E_3$) $\in$ $D(G)$ such that $c^{'}=g_3$ 
\item if $c^{'},c^{''}$ consists of one vertex then there exists no corresponding $5-tuple$ 
%\item if $c^{''}$ consists of one vertex then there exists no  $5-tuple$ for $c{''}$
%\item if $c^{''}$ consists of one vertex then there exists no  $5-tuple$ ($g_4,s_4,P_4,C_4,E_4$) $\in$ $D(G)$ such that $c^{''}=g_4$ 
\end{enumerate}

\end{enumerate}
\end{definition}

\subsection{Induced Subgraph Query}
The improvement algorithms proposed in previous subsection are also efficient for induced subgraph query. In this subsection, we explain the detail algorithms for our induced subgraph query procesing. Many of algorithms of induced subgraph query are same as that of subgraph query, therefore when we reference the previous subsection, take the insides of the parenthesises into consideration to indicate the induced versions.

\subsection{Expansion to Subgraph Isomorphism Query with Improvements}
The method of graph decomposition used in the \textit{Network Algorithm} restricts its  support to induced subgraph isomorphism query only. However, in practice,  subgraph isomorphism query is more likely to suit real world problems therefore we reformulate the \textit{Network Algorithm} to process subgraph isomorphism query. In this subsection, we describe the details of the expansion of the \textit{Network Algorithm} to allow subgraph query processing and also propose some addition improvements. Some of our improvements applied to subgraph query are also applicable to induced subgraph queries which is addressed in the next subsection. For these algorithms, we use parenthesis such as (induced) subgraph to separate the two cases. We first discuss the subgraph isomorphism so ignore the inside of the parentheses throughout this subsection.

\subsubsection{DAG Construction}
The $DAG$ is the main data structure for the database construction and query processing. In this section we outline the steps required to construct 
the $DAG$ as well as the use of the $DAG$ to process queries. First the algorithm for the construction of the DAG is show in Algorithm \ref{alg:alg101} below.


\begin{algorithm}
\caption{createDAG(G)}
\label{alg:alg101}
\begin{algorithmic}
\STATE Input: Model Graphs $G =\{g_1,g_2,\dots,g_m\}$
\STATE Output: $D(G)= \{$($g,s,P,C,E$)$,\ldots,$($g_n,s_n,P_n,C_n,E_n$)$ \}$
\end{algorithmic}
\begin{algorithmic}[1]
\STATE Initialize DAG, $D:=\emptyset$
%\STATE Initialize root, $\emph{ROOT}$ $\leftarrow$ $\emptyset$
\FOR{ $i=1$ to $m$}
 \STATE  $\emph{D} \leftarrow $createTuples($g_i,D$)
\ENDFOR
\RETURN $\emph{D(G)}$
\end{algorithmic}
\end{algorithm}


% \begin{algorithm}
% \caption{DagConstruct(G)}
% \label{alg:alg101}
% \begin{algorithmic}
% \STATE Input: Model Graphs $G$=$\{g_1,g_2,\dots,g_m\}$
%\STATE Output: DAG $D(g)=\{(g,g^{'},g^{''} ,E)\dots (g_n ,g_n^{'},g_{n}^{''} ,E_n )\}$
% \STATE Output: $D(G)$
% \end{algorithmic}
% \begin{algorithmic}[1]
% \STATE Initialize Model Graph Set, $G$ $\leftarrow$ $\emptyset$
% \STATE Initialize root, $\emph{ROOT}$ $\leftarrow$ $\emptyset$
% \STATE Initialize DAG, $\emph{DAG}$ $\leftarrow$ $\emptyset$
% \STATE Initialize model graph count, $\emph{DAGSize}$ $\leftarrow$ $\emptyset$
% \STATE Default node State, $\emph{state}$ $\leftarrow$ $\emph{unsolved}$
% \STATE Default DAG edge $e_{\emptyset}$ $\leftarrow$ $\emptyset$
%\STATE Initialize the largest common subgraph, $S_{max}$=0  
% \STATE Create a $\emph{NULL}$ node as root for $DAG$
% \FOR{ $i=1$ to $m$}
 % \STATE  $\emph{DAG}$ $\leftarrow$ $\emph{DagInsert}$($g_{i},state, parentGraph,c^{'},c^{''},EdgeIDs,DAG$)
 % \STATE  $\emph{DAGSize}$  $\leftarrow$   $\emph{DAGSize + 1}$
% \ENDFOR
% \RETURN $\emph{DAG}$
% \end{algorithmic}
% \end{algorithm}


% \begin{algorithm}
% \caption{DagInsert(g,D)}
% \label{alg:alg102}
% \begin{algorithmic}
% \STATE Input: Model Graph $g$
% \STATE Output: DAG
% \end{algorithmic}
% \begin{algorithmic}[1]
% \IF{ g has  one vertex}
   % \STATE create node for g as child of ROOT
% \RETURN
% \ENDIF
% \FORALL{ ${gm_{1},gm_{2},\dots,gm_{n}}$ in $D$}
    % \IF{ $(gm_{i} \subseteq_{subgraph})$ \AND $(S_{max} < g_{i})$} 
      % \STATE set  $S_{max}:=gm_{i}$
    % \ELSIF{ $S_{max}$ is isomorphic to $gm_{i}$ } 
          % \STATE link $gm_{i}$ to the parent node of $S_{max}$
    % \RETURN
    % \ELSIF{no subgraph found}
       % \STATE $RandomPartition(g,g_{'},g^{''},E)$
       % \STATE $ Addnode(g_{'})$
       % \STATE $ Addnode(g^{''})$
% \ENDIF
% \ENDFOR
% \RETURN
% \end{algorithmic}
% \end{algorithm


% \begin{algorithm}
% \caption{DagInsert($g,D,$)}
% \label{alg:alg102}
% \begin{algorithmic}
% \STATE Input: New Model Graph $g$
% \STATE Output: DAG $D(g)=\{(g,g^{'},g^{''} ,E)\dots (g_n ,g_n^{'},g_{n}^{''} ,E_n )\}$
% \end{algorithmic}
% \begin{algorithmic}[1]
%\STATE $P = \cup_{i=1}^n \{g_i ,g_i^{'},g_i^{''} \}$ where n is the number of Model graphs in DAG
% \STATE $g = \emptyset$
% \STATE $Z= \emptyset$
% \STATE $DAGSize<n$ Number of model graphs already stored in DAG
% \STATE 
% \FOR{k=1 to DAGSize}
    % \STATE $Z$= SubgraphQuery($g_k, g$)
% \ENDFOR
% \IF{ $Z=\emptyset$}
    % \STATE Link $g$ to ROOT node
    % \WHILE{ $|g|>1$}
        % \STATE ($g^{'},g^{''}$)=RandomPartition($g,g^{'},g^{''},E$)
        % \STATE $D$ $\leftarrow$ AddNode($g^{'},D$)
        % \STATE Link $g^{'}$ to $g$ 
        % \STATE $D$ $\leftarrow$ AddNode($g^{''},D$)
        % \STATE Link $g^{''}$ to $g$ 
        % \STATE DagInsert($g^{'},D$)
        % \STATE DagInsert($g^{''},D$)
     % \ENDWHILE
% \ELSIF{$g-|Z|_{max} = \emptyset$ }
% \RETURN
% \ELSE
    % \FORALL{ $g_i$ $\in Z$}
       % \STATE ConnectGraph($g,g_i,g-g_i,E$)
       % \STATE DagInsert($g-g_i$,D)
       % \STATE Link graph ($g-g_i$) to the parent of $g_i$ 
    % \ENDFOR
% \ENDIF
% \RETURN D
% \end{algorithmic}
% \end{algorithm}


\begin{algorithm}
\caption{CreateTuples($g,D$)} 
\label{alg:alg102}
\begin{algorithmic}
\STATE Input: Connected, non-singleton Graph $g$, Current DAG $D$
\STATE Output: DAG with current tuples and new tuples from graph $g$ added, $D^{'}$
\end{algorithmic}
\begin{algorithmic}[1]
%%\STATE $P = \cup_{i=1}^n \{g_i ,g_i^{'},g_i^{''} \}$ where n is the number of Model graphs in DAG
%\STATE Graph set $g \leftarrow \emptyset$
\STATE Isomorphic graph set $Z \leftarrow \emptyset$
\STATE graph $z  \leftarrow \emptyset$
\STATE Default node state $s  \leftarrow unsolved$
\STATE Set of Parent Graphs, $P \leftarrow \emptyset$
\STATE Local DAG initialized with current tuples $D^{'} \leftarrow D$

\FOR{each Tuple $\{(z,s,P_{z},C,E)\}  \in D^{'}$ where vertex count $|z| \leq |g|$}
%\STATE let $z$ be the current graph in DAG
\IF{ graph $z$ is $\subseteq_{(induced)} g$}
\STATE Add graph $z$ to $Z$ 
% \STATE $Z$ $\leftarrow$  SubgraphQuery$(DAGIndexGraph, g)$
\ENDIF
\ENDFOR
\WHILE{$Z\neq \emptyset$} 
\IF{$z=g$}
\STATE $P_{z} \leftarrow P_{g}$
%\STATE let $z$ be a child of $g$'s parents $P_g$ and delete $g$
\STATE $D^{'} \leftarrow D^{'} \cup \{(z,s,P_z ,\emptyset ,\emptyset)\}$
\RETURN $D^{'}$
\ELSE
\FOR{each graph $z$ in $Z$}
\STATE \ $z_{rest}  \leftarrow  (g -_{(induced)} z)$
\IF{$z_{rest}$ is connected}
%\STATE Let \emph{$z_{rest}$} be a child of \emph{$g$}
\STATE $D^{'}  \leftarrow D^{'} \cup \{(g,s,P_g,\emptyset , \emptyset )\}$
\STATE $D^{'}  \leftarrow D^{'} \cup \{(z_{rest},s,\{g\},\emptyset , \emptyset )\}$
\STATE CreateTuples($z_{rest}$)
\RETURN  $D^{'}$
\ENDIF
\ENDFOR
\ENDIF
\ENDWHILE
%\IF{ $g$ is not a singleton graph}
%\IF{$g$ is connected}
\STATE $(z,z_{rest},E) \leftarrow RandomPartition(g)$
%\ELSE
%\STATE distribute the components of $g$ to $z$ and $z_{rest}$, with common Edges $E$
%\ENDIF
%\STATE let $z$ and $z_{rest}$ be the children of $g$
\STATE $D^{'}  \leftarrow D^{'} \cup \{(g,s,\emptyset,\{z,z_{rest}\},E)\}$
\STATE $D^{'}  \leftarrow D^{'} \cup \{(z,s,\{g\},\emptyset,E)\}$
\STATE $D^{'}  \leftarrow D^{'} \cup \{(z_{rest},s,\{g\},\emptyset ,E)\}$
\STATE CreateTuples($z$)
\STATE CreateTuples($z_{rest}$) 
%\ENDIF
\RETURN $D^{'}$
\end{algorithmic}
\end{algorithm}


\begin{algorithm}
\caption{ConnectGraph($g,s,P,\{c^{'},c^{''}\},E$)}
\label{alg:alg103}
\begin{algorithmic}
\STATE Input: graph with disconnected childgraphs ($g,s,P,\{c^{'},c^{''}\},E$)
\STATE Output: graph with connected childgraphs ($g,s,P,\{c^{'},c^{''}\},E$)
\end{algorithmic}
\begin{algorithmic}[1]
\STATE $c^{'}$ = NULL
\STATE $c^{''}$ = NULL
\IF{ graph $g$ has a single edge e($v_1$,$v_2$)}
   \STATE Insert $v_{1}$ to $c^{'}, v_{2}$ to $c^{''}$
\ELSE 
 \WHILE{ $c^{'}$ or $c^{''}$ not connected}
  \IF{ both $c^{'}$ \AND $c^{''}$ are disconnected}
    \STATE move minimum component of the larger $|c^{'},c^{''}|_{max}$ to  $|c^{'},c^{''}|_{min}$
%    \STATE $|c^{'} ,c^{''} |_{max} $ to $|c^{'} ,c^{''} |_{min} $
  \ELSE
    \STATE move component of disconnected graph opposite $c$
  \ENDIF

 \ENDWHILE
\ENDIF
\RETURN ($g,s,P,\{c^{'},c^{''}\},E$)
\end{algorithmic}
\end{algorithm}


\begin{algorithm}
\caption{RandomPartition($g,s,P,\{c^{'},c^{''}\},E$)}
\label{alg:alg104}
\begin{algorithmic}
\STATE Input: graph $g$
\STATE Output: ($g,s,P,\{c^{'},c^{''}\},E$) is a 5-tuple consisting of graph $g$, a state $s$ children $\{c^{'},c^{''}\}$  and a set of edges between the children $E$ 
\end{algorithmic}
\begin{algorithmic}[1]
\IF{ $g$ has a single edge $e=(v_1,v_2)$}
    \STATE insert $v_1$,to $c^{'}$, $v_2$ to $c^{''}$    
 \ELSE
   \STATE RandomPartitionInduced($g,s,P,\{c^{'},c^{''}\},E$)
\ENDIF
\RETURN ($g,s,P,\{c^{'},c^{''}\},E$)
\end{algorithmic}
\end{algorithm}


\begin{algorithm}
\caption{New Network Algorithm, NNA($D , q$)}
\label{alg:alg105}
\begin{algorithmic}
\STATE Input: DAG $D(G)= \{$($g,s,P,\{c^{'},c^{''}\},E$)$,\ldots,$($g_n,s_n,P_n,\{c_n^{'},c_n^{''}\},E_n$)$ \}$, Query $q$
\STATE and the set of all subgraphs in DAG, $G_{all} = \cup_{i=1}^n \{g_1 ,c_1^{'},c_1^{''},g_2 ,c_2^{'},c_2^{''}\ldots \}$ 
\STATE Output: $Z$, the set of all subgraph isomorphisms in Model graphs in $D$ to query $q$ 
\end{algorithmic}
\begin{algorithmic}[1]
\STATE $F \leftarrow \emptyset$
\STATE $Z \leftarrow \emptyset$
\FORALL{ $g,c \in  G_{all}$}
  \STATE set state $s$ to \textit{unsolved}
\ENDFOR
\FOR{ each $g_i \in D$}
   \STATE $F \leftarrow $SubgraphQuery($g_i$,$q$)
   \IF{$F\neq \emptyset$}
      \STATE $Z \leftarrow Z \cup \{F\}$
   \ENDIF
\ENDFOR
\RETURN $Z$
\end{algorithmic}
\end{algorithm}


\begin{algorithm}
\caption{SubgraphQuery($g$, $q$)}
\label{alg:alg106}
\begin{algorithmic}
\STATE Input: Model Subgraph in database  $g$, query $q$, $c^{'}$, $c^{''}$
\STATE Output: Induced Subgraph Isomorphisms, $F$
\end{algorithmic}
\begin{algorithmic}[1]
\IF{ $g$ is unsolved }
    \IF{ $g$ is a singleton }
        \STATE AssignVertex($g$,$q$)
      \ELSE 
         \STATE let $c^{'}$, $c^{''}$ be children of $g$
         \IF{  $c^{'}$ \OR $c^{''}$  have mark dead=true}
          \STATE set $g(dead)=true$
          \ELSIF{SubgraphQuery$(c^{'},q) \neq \emptyset$ \\ or SubgraphQuery$(c^{''} ,q) \neq \emptyset$}
			\STATE set $g(dead)=true$
		\ELSE
			\STATE $F \leftarrow$ Combine(Induced)$(g,q)$
		\ENDIF
	\ENDIF
	\IF{$F \neq \emptyset$}
		\STATE set $g(alive)=true$
	\ELSE
		\STATE set  $g(dead)=true$
	\ENDIF
\ELSE
	\STATE $F$ is the already calculated subgraph isomorphisms from $g$ to $q$
\ENDIF
\RETURN $F$
          
\end{algorithmic}
\end{algorithm}


%%\label{insertion}
CreateTuples(Algorithm\ref{alg:alg102}) outlines the dynamic tuple insertion algorithm for $DAG$. In order to construct $DAG$ with model graphs , we first decompose the model graphs. Then we create and insert the corresponding 5-tuples and the into the $DAG$ to form a graph database. Model graphs are inserted in the $DAG$ sequentially.

In constructing the $DAG$, as a rule, each graph in a tuple directs two (induced) subgraphs. The exception being that $g$ will have no children if $g$ is a singleton graph. The information about the shared vertices in the children and the mappings between a parent and children is  stored with a parent-child relationship in the 5-tuple..

First, the algorithm searches all (induced) subgraphs of $g_i$ in $DAG$(line 6-11). For this (induced) subgraph search, SubgraphQuery (Algorithm\ref{alg:alg6}) explained in the latter section is used. The discovered (induced) subgraphs are inserted to a queue $Z$.

If we find that a graph $z$ in $Z$ is isomorphic to $g_i$, we replace the graph by setting a directed edge from the parent of $g_i$ to $z$ (line 13-16).
The existence of graph isomorphism is easily ensured by checking whether $E(g_i)$ is equals to $E(z)$ or not. In most cases, a (induced) subgraph $z$ of $g_i$ will be discovered. In that case, the algorithm calculates $z_{rest} = g_i -_{(induced)} z$ and $g_i$ is decomposed into the $z$ and the rest graph $z_{rest}$.
If $z_{rest}$ is connected, $z_{rest}$ is inserted to $DAG$ recursively(line 20-26).
There is the case that $z_{rest}$ is disconnected, which is a violation of the requirement that a connected graph is decomposed into two connected graphs. When faced with this violation the algorithm abandons the result and tests the subtractions of all (induced) subgraph isomorphisms of all (induced) subgraphs until $z_{rest}$ becomes connected.

The order in which the  (induced) subgraph are tested for subtraction relies ordering of the graph is $Z$.
We know that The larger a graph is, the more costly it is to compute the (induced) subgraph isomorphisms therefore we sort $Z$ to make larger graphs precede to allow  .

If no (induced) subgraph is discovered or no (induced) subgraph can make $z_{rest}$ connected, $g_i$ is decomposed by the RandomPartition (Algorithm\ref{alg:alg2}).
The decomposed (induced) subgraphs are inserted to $DAG$ recursively(line 36-42).
In the case that $g_i$ is disconnected, $g_i$ is decomposed by distributing its components to two children so that decomposed graphs finally become connected graphs(line 33-35).



The \textit{Network Algorithm} decomposes a graph by distributing vertices randomly and creating subgraphs which induced by the distributed vertices.
This decomposition possibly creates disconnected subgraphs composed of many tiny components.
The number of matchings resulting from a large number of such disconnected graphs to a query graph can easily skyrocket therefore need to make sure that the decomposed subgraphs are connected to avoid this problem. Effectively, the \textit{Network Algorithm} decomposes a graph by cutting edges for induced subgraph query.

We  now  focus on the subgraphs which are not induced. In order to allow a subtraction of an arbitrary subgraph from a graph we have to decompose a graph by sharing vertices.

RandomPartition (Algorithm\ref{alg:alg2}) decomposes a connected graph $g$ into two connected subgraphs.
First, the algorithm distributes edges randomly. If at least one subgraph is disconnected, the two subgraphs transfer their minimal components to the other side.
The transition of a component connects components of the other side because the parent $g$ is connected.
The algorithm repeats the transitions until both subgraphs become connected. Effectively, we are decomposing graphs by sharing vertices.
However, one  cannot decompose a single edge any further by sharing vertices. Therefore the RandomPartition algorithm cannot process a singleton graph or a disconnected graph which includes one or more isolated vertices. We therefore  decompose a single edge in the manner of cutting edges.

\begin{algorithm}
\caption{RandomPartition}
\label{alg:alg2}
\begin{algorithmic}
\STATE Input: Graph $g$
\STATE Output: A pair of graphs $(s_1, s_2)$
\end{algorithmic}
\begin{algorithmic}[1]
\STATE let $s_1 := \emptyset$, $s_2 := \emptyset$
\IF{$g$ is composed of a single edge $e = (v_1, v_2)$}
	\STATE insert $v_1,v_2$ to $s_1,s_2$ respectively
\ELSE
	\STATE $(s_1, s_2)$ RandomPartitionInduced($g$)
	\WHILE{either $s_1$ or $s_2$ is disconnected}
		\IF{both $s_1$ and $s_2$ is disconnected}
			\STATE move minimum component of the larger graph to the other side
		\ELSE
			\STATE move minimum component of the disconnected graph to the other side
		\ENDIF
	\ENDWHILE
\ENDIF
\RETURN $(s_1,s_2)$
\end{algorithmic}
\end{algorithm}

\subsubsection{Query Processing}
There are two types of query processing to the graph database. The induced graph query and the Non induced graph query. In this section we outline the algorithm
 for both types of processing.
Search(Algorithm \ref{alg:alg4}) outlines the procedure for the (induced) subgraph isomorphism query processing.
SubgraphQuery (Algorithm \ref{alg:alg5}) can calculate the (induced) subgraph isomorphisms from any node in $DAG$ to a query graph $q$.
So we can accomplish (induced) subgraph isomorphism query by processing SubgraphQuery with all model graphs as input.

In the procedure of the (induced) subgraph search, all nodes in $DAG$ are supposed to be marked with one of three different tags, $alive$, $dead$ or $unsolved$.
Those tags express the conditions that (a) have been matched already with the query graph and (induced) subgraph isomorphisms exists, (b) have been matched alre
ady and (induced) subgraph isomorphisms do not exist, (c) have not been matched yet respectively.
At the beginning, all nodes in $DAG$ are marked as $unsolved$.

SubgraphQuery works recursively.
If the input graph $g$ have been already marked $alive$ or $dead$, there is nothing to do and the algorithm only returns subgraph isomorphisms $F$ that is supposed to have been already calculated.

If $g$ is a singleton graph, there are no children.
Then, SubgraphQuery calls AssignVertex(Algorithm\ref{alg:alg6}).
AssignVertex simply searches the occurrences of the label of the single vertices in a query graph $q$.

If $g$ marked $unsolved$, we check (induced) subgraph isomorphisms of its children by using SubgraphQuery recursively.
Now we can exploit the useful fact that if a child is not a (induced) subgraph of a query graph, its parent also not a (induced) subgraph of the query graph.
First, in order to remove redundancies we check the tags of children of $g$ before we call SubgraphQuery. 
Next, we call SubgraphQuery for $g$'s children.
If at least one child tagged $dead$ or returns empty sets, we can also assure that $g$ is $dead$  
Only in the case that both of the children return $alive$, we have to calculate the (induced) subgraph isomorphisms from $g$ to $q$ by calling Combine.

\begin{algorithm}[t]
\caption{Search}
\label{alg:alg4}
\begin{algorithmic}
\STATE Input: Query graph $q$, Graph database $D$
\STATE Output: Set of (induced) subgraph isomorphisms $Z$
\end{algorithmic}
\begin{algorithmic}[1]
\STATE $Z := \emptyset$
\STATE mark all nodes in $DAG$ $unsolved$
\FOR{each graph $g \in D$}
	\STATE $F :=$ SubgraphQuery$(g,q)$
	\IF{ $F \neq \emptyset$ }
		\STATE $Z = Z \cup \{F\}$
	\ENDIF
\ENDFOR
\RETURN $Z$
\end{algorithmic}
\end{algorithm}



$DAG$ is designed such that a graph $g$ can make use of its children's (induced) subgraph isomorphisms to $q$ in order to reduce the computational effort of 
calculating (induced) subgraph isomorphisms from $g$ to $q$. For this purpose, a procedure for combining of induced subgraph isomorphisms is required.

Combine (Algorithm\ref{alg:alg11}) is the procedure for the combining subgraph isomorphisms. When Combine is called with $g$ as input, the subgraph 
isomorphisms $F_1, F_2$ from its children $c^{'}, c^{''}$ to $q$ are supposed to be already calculated.
In the case that $g$ is connected, two conditions must be satisfied to combine two functions, $f_1 \in F_1, f_2 \in F_2$.
First, it must be ensured that each corresponding common vertices is mapped correctly onto same vertex(line 11).
Second, the image of non-common vertices of $f_1$ and $f_2$ must be disjoint(line 12).
If these two conditions are not satisfied, the combining of $f_1$ and $f_2$ would not be an injective function.
If $g$ is disconnected, we only need to check one condition that the images of $f_1$ and $f_2$ are disjoint(line 20).
We define the combination of (induced) subgraph isomorphisms as follows
\begin{eqnarray}
\label{eq:eq1}
f(v) = \left\{
\begin{array}{l}
f_1(p_1(v)) (p_1(v) \in V(c^{'}))\\
f_2(p_2(v)) (p_2(v) \in V(c^{''}))
\end{array}
\right.
\end{eqnarray}
 
The Combine algorithm checks all pairs of functions $f_1$ and $f_2$. If a pair satisfies these conditions, the algorithm evaluates and returns it.

However this algorithm cannot process a graph that is not decomposed by shared vertices. As such, if a graph consisting of a single edge
is input, the Combine algorithm calls Combine (induced) (Algorithm\ref{alg:alg8}) which is able to process graphs decomposed in the 
manner of cutting edges.

\begin{algorithm}
\caption{AssignVertex}
\label{alg:alg6}
\begin{algorithmic}
\STATE Input: Singleton graph $g=(V,E,L^v ,L^e ,\mu,\nu)$, Query graph $q=(V_q,E_q,L_q^v ,L_q^e ,\mu_q,\nu_q)$ 
\STATE Output: Subgraph isomorphisms $F$
\end{algorithmic}
\begin{algorithmic}[1]
\STATE $F := \emptyset$
\STATE let $v \in V$
\FOR{each $v_q \in V_q$}
	\IF{$\mu(v)= \mu_q (v_q)$}
		\STATE add $f(v) = v_q$ to $F$
	\ENDIF
\ENDFOR
\RETURN $F$
\end{algorithmic}
\end{algorithm}

\begin{algorithm}
\caption{Combine}
\label{alg:alg11}
\begin{algorithmic}
\STATE Input: Graph $g$, Query graph $q$
\STATE Output: Subgraph isomorphisms $F$
\end{algorithmic}
\begin{algorithmic}[1]
\IF{$g$ is composed of a single edge}
	\STATE $F := $CombineInduced$(g,q)$
\ELSE
	\STATE let $c^{'}$ and $c^{''}$ be children of $g$
	\STATE let $p_1$ and $p_2$ be functions from $g$ to $c^{'}$ and $c^{''}$
	\STATE let $F_1$ and $F_2$ be sets of injection from $c^{'}$ and $c^{''}$ to query graph
	\STATE $F := \emptyset$
	\IF{$g$ is connected}
		\STATE let $V_C \subset V(g)$ be set of shared vertices
		\FOR{each $f_1, f_2$ in $F_1, F_2$}
			\IF{$f_1(p_1(V_C)) = f_2(p_2(V_C))$}
				\IF{$f_1(V(c^{'})) \cap f_2(V(c^{''})) = f_1( p_1(V_C) )$}
					\STATE define subgraph isomorphism $f$ by eq(\ref{eq:eq1})
					\STATE $F = F \cup f $
				\ENDIF
			\ENDIF
		\ENDFOR
	\ELSE
		\FOR{each $f_1 \in F_1, f_2 \in F_2$}
			\IF{$f_1(V(c^{'})) \cap f_2(V(c^{''})) = \emptyset$}
				\STATE define subgraph isomorphism $f$ by eq(\ref{eq:eq1})
				\STATE $F = F \cup f $
			\ENDIF
		\ENDFOR
	\ENDIF
\ENDIF
\RETURN $F$
\end{algorithmic}
\end{algorithm}

\subsection{Expansion to Subgraph Isomorphism Query with Improvements}
The method of graph decomposition used in the \textit{Network Algorithm} restricts its  support to induced subgraph isomorphism query only. However, in practice,  subgraph isomorphism query is more likely to suit real world problems therefore we reformulate the \textit{Network Algorithm} to process subgraph isomorphism query. In this subsection, we describe the details of the expansion of the \textit{Network Algorithm} to allow subgraph query processing and also propose some addition improvements. Some of our improvements applied to induced subgraph query are also applicable to subgraph queries which was addressed in the previous subsection. 

%%\subsection{Induced Subgraph Query}
%The improvement algorithms proposed in previous subsection are also efficient for induced subgraph query.
%In this subsection, we explain the detail algorithms for our induced subgraph query procesing.
%Many of algorithms of induced subgraph query are same as that of subgraph query, therefore when we reference the 
%previous subsection, take the insides of the parenthesises into consideration to indicate the induced versions.

\subsubsection{DAG Construction}
The procedure for the construction of $DAG$ and dynamic insertion algorithm are also almost identical to the case of induced subgraph query.
We now explain the differences. First, a parent-child relationship must store the information about the cut edge instead of the 
shared vertices due of the difference in the manner of decomposition of the graph. Second, the existence of a graph isomorphism is 
assured by checking whether input graph $V(g_i)$ is equal to a graph $V(s)$ in the DAG or not.

In order to decompose an arbitrary graph into two smaller induced subgraphs we have to decompose a graph in the manner of cutting edges
hence we define a procedure for decomposing a randomly connected graph in this manner.

RandomPartitionInduced (Algorithm\ref{alg:alg15}) decomposes a connected graph into two connected induced subgraphs.
First, the algorithm initializes two induced subgraphs by distributed vertices and the edges that the vertices induce.
Then, two subgraphs transfer their minimum components until they become connected, a process similar to the one in the 
algorithm for subgraph query.

The decomposition of a disconnected graph is identical to the one for subgraph query.

\begin{algorithm}
\caption{RandomPartitionInduced}
\label{alg:alg15}
\begin{algorithmic}
\STATE Input: Graph $g$
\STATE Output: A pair of graphs $(s_1, s_2)$
\end{algorithmic}
\begin{algorithmic}[1]
\STATE let $s_1 := \emptyset$, $s_2 := \emptyset$
\STATE distribute vertices of $g$ randomly to $s_1$ or $s_2$ in half
\STATE set edges induced by distributed vertices to $s_1$ and $s_2$ 
\REPEAT
	\IF{both $s_1$ and $s_2$ is disconnected}
		\STATE move minimum component of the larger graph to the other side
	\ELSE
		\STATE move minimum component of the disconnected graph to the other side
	\ENDIF
	\STATE reset edges induced by distributed vertices to $s_1$ and $s_2$ 
\UNTIL{both $s_1$ and $s_2$ is connected}
\RETURN $(s_1,s_2)$
\end{algorithmic}
\end{algorithm}

\subsubsection{Query Processing}
The search processing algorithms for subgraphs, Search(Algorithm\ref{alg:alg4}), SubgraphQuery (Algorithm\ref{alg:alg106}) and AssignVertex(Algorithm\ref{alg:alg6}) 
are also applicable to induced subgraph query. However, due to the difference in the method of decomposition, we need a new procedure for the combining the 
induced subgraph isomorphisms. CombineInduced(Algorithm\ref{alg:alg8}) outlines that procedure.
When CombineInduced is called with graph $g$ as input, the sets of induced subgraph isomorphisms $F_1, F_2$ from its children $c^{'}, c^{''}$ 
to the query graph must have been already calculated, as part recursive calling of \label{alg:alg106}.
There are two conditions that must be satisfied to CombineInduced two functions $f_1 \in F_1 and f_2 \in F_2$.
First, there must exist edges in the query graph correspond to cut edges and vice versa(line 7).
Second, we must ensure that $f_1$ and $f_2$ are disjoint(line 8), otherwise the combination of $f_1$ and $f_2$ will not be injective function.
CombineInduced algorithm checks all pairs of the two functions and returns the sets of its unions that satisfy these conditions.

\begin{algorithm}[t]
\caption{CombineInduced}
\label{alg:alg8}
\begin{algorithmic}
\STATE Input: Graph $g=(V,E,L^v ,L^e ,\mu,\nu)$, Query graph $q=(V_q,E_q,L_q^v ,L_q^e ,\mu_q,\nu_q)$ 
\STATE Output: Subgraph isomorphisms $F$
\end{algorithmic}
\begin{algorithmic}[1]
\STATE let $c^{'}$ and $c^{''}$ be children of $g$
\STATE let $p_1$ and $p_2$ be functions from $g$ to $c^{'}$ and $c^{''}$
\STATE let $F_1$ and $F_2$ be sets of injection from $c^{'}$ and $c^{''}$ to query graph
\STATE $F := \emptyset$
\STATE let $E_c \subset E$ be the set of cut edges
\FOR{each $f_1, f_2$ in $F_1, F_2$}
		\IF{ for each edge $e = (v_1, v_2) \in E_c$ there exists an edge $e_q = (f_1(v_1),f_2(v_2)) \in E_q$ with $\nu(e) = \nu_q(e_q)$ and for each edge $e_q = (u_1, u_2) \in E_q$ such that $u_1 \in f_1(V(c^{'})), u_2 \in f_2(V(c^{''}))$ there exists an edge $e = (f_1^{-1}(u_1), f_2^{-1}(u_2)) \in E_c$ with $\nu_q(e_q) = \nu(e)$}	
			\IF{$f_1(V(c^{'})) \cap f_2(V(c^{''})) = \emptyset$}
				\STATE define subgraph isomorphism $f$ by eq(\ref{eq:eq1})
				\STATE $F = F \cup f $
		\ENDIF
	\ENDIF
\ENDFOR
\RETURN $F$
\end{algorithmic}
\end{algorithm}

\subsection{Network Algorithm with Graph Pruning}
The \textit{Network Algorithm} cannot process a local query efficiently (i.e. In the case that you want to know subgraph isomorphisms for only some graphs in a 
graph database to a query graph, the \textit{Network Algorithm} calculates subgraph isomorphisms for all the nodes in $DAG$).
In contrast our recursive formulation of the \textit{Network Algorithm} only checks the relevant descendants of selected nodes and this allows us to solve 
local queries more efficiently.

One result of the support of local query is that we can make use of some graph pruning techniques. Let us consider a simple example \textit{label check}. 
If a graph database stores the numeric labels of attributed graphs, it may be easily determined when they have no subgraph isomorphisms to 
a query graph just by comparing the attributes of the graph with those of the query graph.
In this way, we can efficiently reduce the candidates graphs for the answer.
Consequently, by pruning graphs in advance, we can significantly reduce the related procesing time.


% LocalWords:  EdgeIDs
