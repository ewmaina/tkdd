We study the problem of scalable enumeration of all subgraph isomorphisms to a query graph that exist in a graph database. We  propose several improvements and extensions to known algorithms to solve this problem that result in order of magnitude improvement in scalability. 

Graphs have become increasingly important structures for representing and understanding complex structures. Deciding whether a graph is a subgraph of another graph is called the subgraph isomorphism problem. In this study we are concerned with the retrieval of all subgraph isomorphisms to a query that are present in a graph database.
We define the retrieval of isomorphic subgraphs to satisfy two conditions.  First, a decision whether or not a graph isomorphic to a subgraph of a query graph exists in the graph database. Second, the actual enumeration of all graphs present in the database isomorphic to subgraphs of the query. However the retrieval of all subgraph isomorphisms is complicated and expensive because the subgraph isomorphism problem is NP-complete.
Recently, there have been several studies on efficiency of retrieval of subgraph isomorphisms from a graph database but few have investigated decomposition methodology to reduce the larger, difficult graph matching problem to smaller more managable sub-problems in a divide and conquer fashion. 

%%Messmer et al. studied an interesting method for retrieval of induced subgraph isomorphisms of a query graph. They proposed a so-called \textit{Network Method} that stores graphs  in a network structure  by recursively decomposing the individual graphs in a divide and conquer fashion, and the algorithms to facilitate retrieval using the structure.

In this paper study the problem of efficient enumeration of all subgraph isomorphisms of database graphs that exist in a query graph. Starting with Directed Acyclic Graph (DAG) data structure to represent decomposed subgraphs, we formulate a fast algorithm for decomposing and searching for subgraphs in large database scalably.

% for the \textit{Network Method} and extend it to cover retrieval of non-induced subgraph isomorphisms of a query graph. 
We propose a new algorithm which recursively decomposes random graphs and immediately rearranges fragmented products so that only two subgraphs. This subgraph decomposition without fragmentation process is efficient for both positive (due to compact representation) and non-positive queries (due to minimal computation). The DAG data structure amplifies this properties.

% \textit{Fast  Network Method}
%%recombining and reordering of the graph fragments to restrict decomposed graph count and maximize decomposed graph size. 

We present experimental results where we compare our proposed algorithms with two well known algorithms: Messmer's all subgraph isomorphism detection Algorithm that employs a DAG data structure for storage of subgraphs in a database and the isomorphism detection only algorithm, VF2, which is the state-of-the-art for sequential one-on-one isomorphism testing. The results show that the proposed improvements result in an order of magnitude improvement in scalability over the original \textit{Network Method}  for query graphs of up to 500 nodes to a database of 20,000 graphs. We also show a substantial improvement over the VF2 algorithm despite the increased workload of detecting all subgraph isomorphisms. Our method is particularly suited to larger query graphs or very larger graph databases.

%Graphs have become increasingly ubiquitous and very important structures for representing and understanding complex structures. In many cases the decomposition of Graphs and their representation has become central problem affecting the tractability of graph search algorithms. Traditional graph search 
