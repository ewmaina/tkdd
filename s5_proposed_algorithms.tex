In this section, we present details for the proposed \textit{ New Network algorithm} and related algorithms. 
There are two main contributions: a new faster search strategy and extension to processing of subgraph queries.
Further, we also describe in detail the improved algorithms for processing of induced subgraph query.

%We now describe the algorithms for processing subgraph query and induced subgraph query in detail.

As an  overview, the algorithms solve two problems, database creation and query processing. Accordingly we divide the discussion into them into two corresponding parts.
The first part is concerned with the construction of a dynamic acyclic graph $DAG$ to store the model graphs. 
The model graphs are stored in a network with their decomposed subgraphs and the related links and state information. 
Specifically, we construct a $DAG$ consisting of a set of 5-Tuples, each of which  stores information about a graph, its state, its parents, its children and the Edges connecting the pair of child graphs. 
Creation of the DAG is performed by recursively decomposing the Model graphs and creating a 5-tuple for each resulting graph to store the information. 
The 5-tuples can be considered as nodes in the DAG as they are connected by a directed edge from parent to child.

The second group of Algorithms concerns the process of detecting subgraph isomorphisms using $DAG$ constructed in previous part. 
The $DAG$ is the central data structure for this work and is considered  a global variant in all the algorithms. 

\subsection{DAG definition}
First we make a formal definition to help us write the algorithms for the DAG more concisely.


Let $G=\{g_1 ,\ldots, g_n \}$ be a set of model graph. 
\begin{definition}
A \emph{DAG}  $D(G)$  is a finite set of 5-tuple ($g,s,P,C,E$) constructed by decomposition of a set of model graphs $G$ where $g$ is a graph, $s$ is one of $\{unsolved, dead, alive\}$,  $P$ is a set of one or more parent graphs $\{p^{'},p^{''}\ldots \}$ , $C$ is a set of two child graphs $\{c^{'},c^{''}\}$ of graph $g$, and $E$ is a set of edges connecting the child graphs $c{'}$ and $c^{''}$. 
The $DAG$ has to satisfy the following conditions.


\begin{enumerate}[(1)]
%\item $g$ and $c^{'} ,c^{''} $ are graphs with $c^{'}\subset g$ and $c^{''} \subset g$.
\item  $c^{'}\subset g$ and $c^{''} \subset g$ for each child  $c^{'},c^{''}$ of graph $g$
\item  $g \subset p^{'}$, $g \subset p^{''} \ldots$, for each parent in $p$ of graph $g$
%\item $E$ is a set of edges such that $g=c^{'}\cup_{E} c^{''}$
\item $g=c^{'}\cup_{E} c^{''}$
%\item For each $g_{i}$ in $G$ there exists a $5-tuple$ ($g,s,P,C,E$) $\in$ $D(G)$ 
\item For each $g_{i}$ in $G$ there exists a $5-tuple$ ($g,s,P,C,E$) $\in$ $D(G)$ 
\item For each  $5-tuple$ ($g,s,P,C,E$) $\in$ $D(G)$ there exists no other  $5-tuple$ ($g,s,P,C,E$) $\in$ $D(G)$ with $g=g_1$.
\item For each  $5-tuple$ ($g,s,P,C,E$) $\in$ $D(G)$

\begin{enumerate}[a.]
\item if $c^{'},c^{''} \in C$  consist of graphs with more than one vertex, then there exists a corresponding pair of  $5-tuple$ 
\item if $c^{'},c^{''} \in C$ consist of one vertex then there exists no corresponding $5-tuple$ 
\end{enumerate}

\end{enumerate}
\end{definition}

\begin{definition}
A graph $g$ is disconnected if its vertex set $V$ can be partitioned into two nonempty, disjoint subsets $V_1$ and $V_2$ such that there exists no edge in $g$ whose one end is in subset $V_1$ and the other end is in subset $V_2$
\end{definition}

\begin{definition}
The size of a graph $g$ is defined as $|V| + |E|$, the sum of the number of Vertices and Edges  in the graph, and is denoted by $|g|$. 
\end{definition}  

\begin{definition}
The \textit{minimal component} of a disconnected graph $g$ is defined as a  disjoint component graph of a disconnected graph such that no other disjoint component has fewer vertices.
The \text{minimal component} of disconnected graph $g$ is denoted by $g_{min}$.
\end{definition}  


\subsection{Expansion to Subgraph Isomorphism Query with reduced search space}

The method of graph decomposition used in the \textit{Network Algorithm} is restricted in its  support to induced subgraph isomorphism query processing only. 
However, in practice,  subgraph isomorphism query is more likely to suit real world problems therefore we reformulate the \textit{Network Algorithm} to process subgraph isomorphism query. 
In this subsection, we describe the details of the expansion of the \textit{Network Algorithm} to allow subgraph query processing and also propose some addition improvements. 
Some of our improvements applied to subgraph query are also applicable to induced subgraph queries which is addressed in the next subsection. For these algorithms, we use parenthesis such as (induced) subgraph to separate the two cases. 
We first discuss the subgraph isomorphism so ignore the inside of the parentheses throughout this subsection.

\subsubsection{DAG Construction}
The $DAG$ is the main data structure for the database construction and query processing. 
In this section we outline the steps required to construct the $DAG$ as well as the use of the $DAG$ to process queries. 
First the algorithm for the construction of the DAG is show in Algorithm \ref{alg:alg01} below.


\begin{algorithm}
\caption{createDAG(G)}
\label{alg:alg01}
\begin{algorithmic}
\STATE Input: Model Graphs $G =\{g_1,g_2,\dots,g_m\}$
\STATE Output: $D(G)= \{$($g,s,P,C,E$)$,\ldots,$($g_n,s_n,P_n,C_n,E_n$)$ \}$
\end{algorithmic}
\begin{algorithmic}[1]
\STATE Initialize DAG, $D:=\emptyset$
\FOR{ $i=1$ to $m$}
 \STATE  $\emph{D} \leftarrow $createTuples($g_i,D$)
\ENDFOR
\RETURN $\emph{D(G)}$
\end{algorithmic}
\end{algorithm}

\begin{algorithm}
\caption{CreateTuples($g,D$)} 
\label{alg:alg02}
\begin{algorithmic}
\STATE Input: Connected, non-singleton Graph $g$, Current DAG $D$
\STATE Output: DAG with current tuples and new tuples from graph $g$ added, $D^{'}$
\end{algorithmic}
\begin{algorithmic}[1]
%%\STATE $P = \cup_{i=1}^n \{g_i ,g_i^{'},g_i^{''} \}$ where n is the number of Model graphs in DAG
%\STATE Graph set $g \leftarrow \emptyset$
\STATE Isomorphic graph set $Z \leftarrow \emptyset$
\STATE graph $ z  \leftarrow \emptyset $
\STATE Default node state $s  \leftarrow unsolved$
\STATE Set of Parent Graphs, $P \leftarrow \emptyset$
\STATE Local DAG initialized with current tuples $D^{'} \leftarrow D$

\FOR{each Tuple $\{(z,s_{z} ,P_{z},\{c^{'}, c^{''}\},E)\}  \in D^{'}$ where vertex count $|z| \leq |g|$}
%\STATE let $z$ be the current graph in DAG
\IF{ graph $z$ is $\subseteq_{(induced)} g$}
\STATE Add graph $z$ to $Z$ 
% \STATE $Z$ $\leftarrow$  SubgraphQuery$(DAGIndexGraph, g)$
\ENDIF
\ENDFOR
\WHILE{$Z\neq \emptyset$} 
\IF{$z=g$}
\STATE $P_{z} \leftarrow P_{g}$
%\STATE let $z$ be a child of $g$'s parents $P_g$ and delete $g$
\STATE $D^{'} \leftarrow D^{'} \cup \{(z,s,P_z ,\{\emptyset \} ,\emptyset)\}$
\RETURN $D^{'}$
\ELSE
\FOR{each graph $z$ in $Z$}
\STATE \ $z_{rest}  \leftarrow  (g -_{(induced)} z)$
\IF{$z_{rest}$ is connected}
%\STATE Let \emph{$z_{rest}$} be a child of \emph{$g$}
\STATE $D^{'}  \leftarrow D^{'} \cup \{(g,s_{g},P_g,\{ \emptyset \} , \emptyset )\}$
\STATE $D^{'}  \leftarrow D^{'} \cup \{(z_{rest},s_{z_{rest}},\{g\},\{\emptyset \} , \emptyset )\}$
\STATE CreateTuples($z_{rest}$)
\RETURN  $D^{'}$
\ENDIF
\ENDFOR
\ENDIF
\ENDWHILE
%\IF{ $g$ is not a singleton graph}
%\IF{$g$ is connected}
\STATE  $(g,s_{g},\emptyset,\{z,z_{rest}\},E) \leftarrow RandomPartition(g,s_{g},\emptyset,\{\emptyset \}, \emptyset )$
%\ELSE
%\STATE distribute the components of $g$ to $z$ and $z_{rest}$, with common Edges $E$
%\ENDIF
%\STATE let $z$ and $z_{rest}$ be the children of $g$
\STATE $D^{'}  \leftarrow D^{'} \cup \{(g,s_{g},\emptyset,\{z,z_{rest}\},E)\}$
\STATE $D^{'}  \leftarrow D^{'} \cup \{(z,s_{z},\{g\},\{ \emptyset \}, \emptyset )\}$
\STATE $D^{'}  \leftarrow D^{'} \cup \{(z_{rest},s_{z_{rest}},\{g\}, \{ \emptyset \} , \emptyset )\}$
\STATE CreateTuples($z$)
\STATE CreateTuples($z_{rest}$) 
%\ENDIF
\RETURN $D^{'}$
\end{algorithmic}
\end{algorithm}


\begin{algorithm}
\caption{ConnectGraph($g,s,P,\{c^{'},c^{''}\},E$)}
\label{alg:alg03}
\begin{algorithmic}
\STATE Input: graph with disconnected childgraphs ($g,s,P,\{c^{'},c^{''}\},E$)
\STATE Output: graph with connected childgraphs ($g,s,P,\{c^{'},c^{''}\},E$)
\end{algorithmic}
\begin{algorithmic}[1]
\STATE $c^{'} \leftarrow  \emptyset $ 
\STATE $c^{''} \leftarrow  \emptyset $ 
\IF{ graph $g$ has a single edge e($v_1$,$v_2$)}
   \STATE Insert $v_{1}$ to $c^{'}, v_{2}$ to $c^{''}$
\ELSE 
\REPEAT
%%\STATE $E_{c^{'}} \leftarrow \emptyset $
%%\STATE $E_{c^{''}} \leftarrow \emptyset $
	\IF{ $c^{'}$ \AND $c^{''}$ are disconnected}
             \IF{ $|c^{'}| > |c^{''}|$ }
                  \STATE $  c^{''} \leftarrow c^{''} \cup  c^{'}_{min} $
             \ELSE
                \STATE $  c^{'} \leftarrow c^{'} \cup c^{''}_{min} $
%%              \STATE $min \{c^{'} , c^{''} \} \leftarrow min \{c^{'} , c^{''} \} \cup $ minimal component of $max\{c^{'} , c^{''} \}$ 
%%		\STATE select larger graph from $\{c^{'} , c^{''} \}$ 
%%              \STATE move minimal component to opposite graph
             \ENDIF
        \ELSE
                \IF { $c^{'}$ is disconnected}
                   \STATE $  c^{''} \leftarrow c^{''} \cup c^{'}_{min} $
%                  \STATE select disconnected graph from $\{c^{'} , c^{''} \}$
%                  \STATE move minimal component to opposite graph
                \ELSE
                   \STATE $  c^{'} \leftarrow c^{'} \cup c^{''}_{min}$ 
                \ENDIF
        \ENDIF
%%	\STATE reset edges induced by distributed vertices to $c^{'}$ and $c^{''}$ 

\UNTIL{ $c^{'}$ \AND $c^{''}$ are connected}
%% \WHILE{ $c^{'}$ or $c^{''}$ not connected}
%%  \IF{ both $c^{'}$ \AND $c^{''}$ are disconnected}
%%    \STATE select graph $|c^{'},c^{''}|_{max}$
%%    \STATE move minimum component to $|c^{'},c^{''}|_{min}$
%%  \ELSE
%%    \STATE select disconnected component from $\{c^{'} , c^{''} \}$
%%    \STATE move component of disconnected graph to opposite $c$
%%  \ENDIF

%% \ENDWHILE
\ENDIF
\RETURN ($g,s,P,\{c^{'},c^{''}\},E$)
\end{algorithmic}
\end{algorithm}


\begin{algorithm}
\caption{RandomPartition($g,s,P,\{c^{'},c^{''}\},E$)}
\label{alg:alg04}
\begin{algorithmic}
\STATE Input: graph $g$
\STATE Output: ($g,s,P,\{c^{'},c^{''}\},E$) is a 5-tuple consisting of graph $g$, a state $s$ children $\{c^{'},c^{''}\}$  and a set of edges between the children $E$ 
\end{algorithmic}
\begin{algorithmic}[1]
\IF{ $g$ has a single edge $e=(v_1,v_2)$}
    \STATE   $c^{'} \leftarrow \{v_1 \}$
    \STATE   $c^{''} \leftarrow \{v_2 \}$    
 \ELSE
   \STATE ($g,s,P,\{c^{'},c^{''}\},E$) $\leftarrow$ RandomPartitionInduced($g,s,P,\{\emptyset \},E$)
\ENDIF
\RETURN ($g,s,P,\{c^{'},c^{''}\},E$)
\end{algorithmic}
\end{algorithm}


\begin{algorithm}
\caption{New Network Algorithm, NNA($D , q$)}
\label{alg:alg05}
\begin{algorithmic}
\STATE Input: Model graphs in DAG, $D(G)= \{$($g_1 ,s_1 ,P_1 ,\{c^{'}_1 ,c^{''}_1 \},E$)$,\ldots,$($g_n,s_n,P_n,\{c_n^{'},c_n^{''}\},E_n$)$ \}$, Query $q$
\STATE and the set of all subgraphs in DAG, $G_{all} = \cup_{i=1}^n \{g_1 ,c_1^{'},c_1^{''},g_2 ,c_2^{'},c_2^{''}\ldots \}$ 
\STATE Output: $Z$, the set of all subgraph isomorphisms in Model graphs in $D$ to query $q$ 
\end{algorithmic}
\begin{algorithmic}[1]
\STATE $F \leftarrow \emptyset$
\STATE $Z \leftarrow \emptyset$
%%\FOR{ each $(g,s,P,\{c^{'},c^{''}\}) \in  D$}
\FOR{ each subgraph in $G_{all}$}
  \STATE  $s \leftarrow unsolved$
\ENDFOR
\FOR{ each $(g,s,P,\{c^{'},c^{''}\}) \in D$}
   \STATE $F \leftarrow $SubgraphQuery($g,s,P,\{c^{'},c^{''}\},q$)
   \IF{$F\neq \emptyset$}
      \STATE $Z \leftarrow Z \cup \{F\}$
   \ENDIF
\ENDFOR
\RETURN $Z$
\end{algorithmic}
\end{algorithm}


\begin{algorithm}
\caption{SubgraphQuery$(g,s,P,\{c^{'},c^{''}\},E, q)$}
\label{alg:aalg06}
\begin{algorithmic}
\STATE Input: Model graph 5-Tuple ($g,s,P,\{c^{'},c^{''}\},E$), query $q$
\STATE Output: Induced Subgraph Isomorphisms, $F$
\end{algorithmic}
\begin{algorithmic}[1]
%%\STATE $F \leftarrow \emptyset $
\IF{ $s$ = $unsolved$ }
\STATE $F \leftarrow \emptyset $
    \IF{ $g$ is a singleton }
        \STATE $F \leftarrow$ AssignVertex($g$,$q$)
      \ELSE 
        %% \STATE let $c^{'}$, $c^{''}$ be children of $g$
         \IF{  $s_{c^{'}}  = dead$ \OR $s_{c^{''}} = dead$}
          \STATE  $s \leftarrow dead$
          \RETURN $F$
          \ELSIF{SubgraphQuery$(c^{'} ,s_{c^{'}} ,P_{c^{'}} ,\{c^{'}_1 ,c^{''}_1 \},E_1 ,q) = \emptyset$ \\ \OR SubgraphQuery$(c^{''} ,s_{c^{''}} ,P_{c^{''}} ,\{c^{'}_2 ,c^{''}_2 \},E_2 ,q) = \emptyset$}
			\STATE $s \leftarrow dead$
                         \RETURN $F$
		\ELSE
			\STATE $F \leftarrow$ CombineInduced$(g,s,P,\{c^{'},c^{''}\},E ,q)$
		\ENDIF
	\ENDIF
	\IF{$F \neq \emptyset$}
		\STATE $s \leftarrow alive$
	\ELSE
		\STATE $s \leftarrow dead$
                 \RETURN $F$
	\ENDIF
\ELSE
%%	\STATE $F$ is the already calculated subgraph isomorphisms from $g$ to $q$
        \RETURN $F$
\ENDIF
%%\RETURN $F$
          
\end{algorithmic}
\end{algorithm}


CreateTuples(Algorithm\ref{alg:alg02}) outlines the dynamic tuple insertion algorithm for $DAG$. 
In order to construct $DAG$ with model graphs , we first decompose the model graphs. 
Then we create and insert the corresponding 5-tuples and the into the $DAG$ to form a graph database. 
Model graphs are inserted in the $DAG$ sequentially.

In constructing the $DAG$, as a rule, each graph in a tuple directs two (induced) subgraphs. 
The exception being that $g$ will have no children if $g$ is a singleton graph. 
The information about the shared vertices in the children and the mappings between a parent and children is  stored with a parent-child relationship in the 5-tuple.

First, the algorithm searches all (induced) subgraphs of $g_i$ in $DAG$(line 6-10). 
For this (induced) subgraph search, SubgraphQuery (Algorithm \ref{alg:aalg06}) explained in the latter section is used. 
The discovered (induced) subgraphs are inserted to a queue $Z$.

If we find that a graph $z$ in $Z$ is isomorphic to $g_i$, we replace the graph by setting a directed edge from the parent of $g_i$ to $z$ (line 12-15).
The existence of graph isomorphism is easily ensured by checking whether $E(g_i)$ is equals to $E(z)$ or not. 
In most cases, a (induced) subgraph $z$ of $g_i$ will be discovered. 
In that case, the algorithm calculates $z_{rest} = g_i -_{(induced)} z$ and $g_i$ is decomposed into the $z$ and the rest graph $z_{rest}$.
If $z_{rest}$ is connected, $z_{rest}$ is inserted to $DAG$ recursively(line 17-25).
There is the case that $z_{rest}$ is disconnected, which is a violation of the requirement that a connected graph is decomposed into two connected graphs. 
When faced with this violation the algorithm abandons the result and tests the subtractions of all (induced) subgraph isomorphisms of all (induced) subgraphs until $z_{rest}$ becomes connected.

The order in which the  (induced) subgraph are tested for subtraction relies ordering of the graph in $Z$.
We know that The larger a graph is, the more costly it is to compute the (induced) subgraph isomorphisms therefore we sort $Z$ to make larger graphs precede to ensure larger subgraphs are shared.

If no (induced) subgraph is discovered or no (induced) subgraph can make $z_{rest}$ connected, $g_i$ is decomposed by the RandomPartition (Algorithm\ref{alg:alg04}).
The decomposed (induced) subgraphs are inserted to $DAG$ recursively(line 38-33).
%In the case that $g_i$ is disconnected, $g_i$ is decomposed by distributing its components to two children so that decomposed graphs finally become connected graphs(line 33-35).

The \textit{Network Algorithm} decomposes a graph by distributing vertices randomly and creating subgraphs which are induced by the distributed vertices.
This decomposition may create disconnected subgraphs composed of many tiny components.
Conceivably, the number of matchings resulting from a large number of such disconnected graphs to a query graph can easily skyrocket hence need to make sure that the decomposed subgraphs are connected to avoid this problem. Effectively, the \textit{Network Algorithm} decomposes a graph by partitioning vertices for induced subgraph query

We now focus on processing of subgraphs which are not induced. In this case, the subtraction of an arbitrary subgraph from a graph requires the decomposition of a graph by partitioning edges.

RandomPartition (Algorithm\ref{alg:alg04}) decomposes a connected graph $g$ into two connected subgraphs.
First, the algorithm distributes edges at random to form two graphs. If at least one subgraph is disconnected, the two subgraphs transfer their minimal components to the opposite side. 
This movement of components likely results in connected graphs because the parent graph $g$ is connected.
The transfer of components is repeated until both subgraphs become connected. 

The limit for decomposing a graph comes when there is a single edge, which we cannot decompose further by partitioning vertices. 
Hence, RandomPartition algorithm cannot process a singleton graph or a disconnected graph which includes one or more isolated vertices. 
To overcome this limitation, we decompose a single edge by partitioning vertices.

%% \begin{algorithm}
%% \caption{RandomPartition($g$)}
%% \label{alg:alg07}
%% \begin{algorithmic}
%% \STATE Input: Graph $g$
%% \STATE Output: A pair of graphs $(s_1, s_2)$
%% \end{algorithmic}
%% \begin{algorithmic}[1]
%% \STATE let $s_1 := \emptyset$, $s_2 := \emptyset$
%% \IF{$g$ is composed of a single edge $e = (v_1, v_2)$}
%% 	\STATE insert $v_1,v_2$ to $s_1,s_2$ respectively
%% \ELSE
%% 	\STATE $(s_1, s_2)$ RandomPartitionInduced($g$)
%% 	\WHILE{either $s_1$ or $s_2$ is disconnected}
%% 		\IF{both $s_1$ and $s_2$ is disconnected}
%% 			\STATE select graph $|s_{1} ,s_{2} |_{max}$
%%                         \STATE move minimum component to opposite side
%% 		\ELSE
%% 			\STATE select disconnected graph from $\{s_1 , s_2 \}$
%%                         \STATE move minimum component to opposite side
%% 		\ENDIF
%% 	\ENDWHILE
%% \ENDIF
%% \RETURN $(s_1,s_2)$
%% \end{algorithmic}
%% \end{algorithm}

\subsubsection{Query Processing}
There are two types of query processing to the graph database; induced subgraph query and the subgraph query. 
In this section we outline the algorithms and discuss both types of processing.
The \textit{New Network Algorithm} (Algorithm \ref{alg:alg05}) outlines the procedure for the (induced) subgraph isomorphism query processing.
Using the SubgraphQuery (Algorithm \ref{alg:aalg06}) we can calculate the (induced) subgraph isomorphism of any graph stored in a node in $DAG$ to a query graph $q$.
We evaluate (induced) subgraph isomorphism query by processing SubgraphQuery with all model graphs as input.

In the procedure for the (induced) subgraph search, all nodes in $DAG$ are set to one of three states using tags, $alive$, $dead$ or $unsolved$.
Those tags indicate three states:(a) already matched  with the query graph and (induced) subgraph isomorphisms exists, (b) already matched and (induced) subgraph isomorphism does not exist, (c) not been matched yet respectively. Initially, all nodes in $DAG$ are tagged $unsolved$.

SubgraphQuery processes the decomposed input graph in the $DAG$ in a recursive manner.
If the input graph $g$ is already tagged $alive$ or $dead$, there is nothing to do and the algorithm only returns subgraph isomorphism $F$  already calculated.

If $g$ is a singleton graph there are no child graphs so SubgraphQuery calls AssignVertex(Algorithm\ref{alg:alg07}) to simply search the occurrences of the label of the single vertex in a query graph $q$.

If $g$ marked $unsolved$, we check (induced) subgraph isomorphism of its children by calling SubgraphQuery recursively.
We exploit the knowledge that if a child is not a (induced) subgraph of a query graph, its parent cannot be a (induced) subgraph of the query graph.
First, in order to avoid unnecessary isomorphism computation we check the tags of children of $g$ before we call SubgraphQuery. 
Next, we call SubgraphQuery for $g$'s children. If at least one child tagged $dead$ or returns an empty set, we are can tag $g$ as $dead$.  
Only in the case that both of the children are tagged with or return the state $alive$ do we have to calculate the (induced) subgraph isomorphisms from $g$ to $q$.

To process the query $q$, we can use already computed child graph's (induced) subgraph isomorphism to $q$ to reduce the computational effort of 
calculating (induced) subgraph isomorphism from $g$ to $q$. For this purpose, a procedure for combining of induced subgraph isomorphisms is required.

Combine (Algorithm\ref{alg:alg08}) is the procedure for the combining subgraph isomorphisms. When Combine is called with $g$ as input, the subgraph 
isomorphisms $F_1, F_2$ from its children $c^{'}, c^{''}$ to $q$ are supposed to be already computed.
In the case that $g$ is connected, two conditions must be satisfied to combine two functions, $f_1 \in F_1, f_2 \in F_2$.
First, it we must ensure that each corresponding common vertices is mapped correctly onto same vertex (line 11).
Second, the image of non-common vertices of $f_1$ and $f_2$ must be disjoint (line 12).
Satisfaction of these conditions ensures that the combining of $f_1$ and $f_2$ is injective.
If $g$ is disconnected, we only need to check one condition that the images of $f_1$ and $f_2$ are disjoint (line 20).
We define the combination of (induced) subgraph isomorphisms as follows
\begin{eqnarray}
\label{eq:eq1}
f(v) = \left\{
\begin{array}{l}
f_1(\bar{p_1}(v)) \quad \bar{p_1}(v) \in V_{c^{'}}\\
f_2(\bar{p_2}(v)) \quad \bar{p_2}(v) \in V_{c^{''}}
\end{array}
\right.
\end{eqnarray}
 
Where $\bar{p_1}$ and $\bar{p_2}$ are functions from parent graph $g$ to childgraphs $c^{'}$ and $c^{''}$ respectively.
The Combine algorithm checks all pairs of functions $f_1$ and $f_2$ and when a pair is found that satisfies these conditions, the algorithm evaluates and returns it.

However this algorithm is limited to graphs decomposable by partitioning edges such that if a graph consisting of a single edge
is input, the Combine algorithm has to call Combineinduced (Algorithm\ref{alg:alg010}) which is able to process graphs decomposable by partitioning vertices..

\begin{algorithm}
\caption{AssignVertex($g,q$)}
\label{alg:alg07}
\begin{algorithmic}
\STATE Input: Singleton graph $g=(V,E,L^v ,L^e ,\mu,\nu)$, Query graph $q=(V_q,E_q,L_q^v ,L_q^e ,\mu_q,\nu_q)$ 
\STATE Output: Subgraph isomorphisms $F$
\end{algorithmic}
\begin{algorithmic}[1]
\STATE $F \leftarrow \emptyset$
\STATE let $v \in V$
\FOR{each $v_q \in V_q$}
	\IF{$\mu(v)= \mu_q (v_q)$}
		\STATE add $f(v) = v_q$ to $F$
	\ENDIF
\ENDFOR
\RETURN $F$
\end{algorithmic}
\end{algorithm}

\begin{algorithm}
\caption{Combine$(g,s,P,\{c^{'},c^{''}\},E, q)$}
\label{alg:alg08}
\begin{algorithmic}
\STATE Input: Graph 5-Tuple $(g,s,P,\{c^{'},c^{''}\},E)$, Query graph $q$
\STATE Output: Subgraph isomorphisms $F$
\end{algorithmic}
\begin{algorithmic}[1]
\IF{$g$ is composed of a single edge}
	\STATE $F \leftarrow$ CombineInduced$(g,s,P,\{c^{'},c^{''}\},E, q)$
\ELSE
%%	\STATE let $c^{'}$ and $c^{''}$ be children of $g$
        \STATE $F_1 \leftarrow$ Combine$(c^{'} , s_{c^{'}} , P_{c^{'}} , \{c^{'}_1 ,c^{''}_1 \},E_1 , q)$
        \STATE $F_2 \leftarrow$ Combine$(c^{''} ,s_{c^{''}} ,P_{c^{''}} ,\{c^{'}_2 ,c^{''}_2 \},E_2 , q)$
%%	\STATE let $p_1$ and $p_2$ be functions from $g$ to $c^{'}$ and $c^{''}$
%%	\STATE let $F_1$ and $F_2$ be sets of injection from $c^{'}$ and $c^{''}$ to query graph
	\STATE $F \leftarrow \emptyset$
	\IF{$g$ is connected}
		\STATE $V_c = V_g - (V_1 \cup V_2) $ 
		\FOR{each pair ($f_1, f_2$) in ($F_1, F_2$) }
			\IF{$f_1(\bar{p_1}(V_c)) = f_2(\bar{p_2}(V_c))$}
				\IF{$f_1(V_{c^{'}}) \cap f_2(V_{c^{''}}) = f_1( \bar{p_1}(V_c) )$}
%%					\STATE define subgraph isomorphism $f$ by eq(\ref{eq:eq1})
                                        \STATE $g \leftarrow (c^{'} \cup_{E} c^{''})$
                                        \STATE $V_g \leftarrow f(V_{c^{'}} \cup V_{c^{''}} })$
                                        \STATE $F \leftarrow  F \cup \{ f \} $
				\ENDIF
			\ENDIF
		\ENDFOR
	\ELSE
		\FOR{each $f_1 \in F_1, f_2 \in F_2$}
			\IF{$f_1(V_{c^{'}}) \cap f_2(V_{c^{''}}) = \emptyset$}
%%				\STATE define subgraph isomorphism $f$ by eq(\ref{eq:eq1})
                                \STATE $g \leftarrow (c^{'} \cup_{E} c^{''})$
                                \STATE $V_g \leftarrow f(V_{c^{'}} \cup V_{c^{''}} })$
                                \STATE $F \leftarrow  F \cup \{ f \} $
			\ENDIF
		\ENDFOR
	\ENDIF
\ENDIF
\RETURN $F$
\end{algorithmic}
\end{algorithm}

%\subsection{Expansion to Subgraph Isomorphism Query with Improvements}
%The method of graph decomposition used in the \textit{Network Algorithm} restricts its  support to induced subgraph isomorphism query only. However, in practice,  subgraph isomorphism query is more likely to suit real world problems therefore we reformulate the \textit{Network Algorithm} to process subgraph isomorphism query. In this subsection, we describe the details of the expansion of the \textit{Network Algorithm} to allow subgraph query processing and also propose some addition improvements. Some of our improvements applied to induced subgraph query are also applicable to subgraph queries which was addressed in the previous subsection. We first discuss the subgraph isomorphism so ignore the inside of the parentheses throughout this subsection.

\subsection{Induced Subgraph Query}
The improvement algorithms proposed in previous subsection are also efficient for induced subgraph query.
In this subsection, we explain the detail algorithms for our induced subgraph query processing.
Many of algorithms of induced subgraph query are same as that of subgraph query, therefore when we reference the 
previous subsection, take the insides of the parenthesises into consideration to indicate the induced versions.

\subsubsection{DAG Construction}
The procedure for the construction of $DAG$ and dynamic insertion algorithm are also almost identical to the case of subgraph query.
We now explain the differences. First, a parent-child relationship must store the information about the edges between partitioned vertices instead of the vertices common graphs. 
This is due of the difference in the manner of decomposition of the graphs. 
Second, the existence of a graph isomorphism is assured by checking whether input graph $V(g_i)$ is equal to a graph $V(s)$ in the DAG or not.

In order to decompose an arbitrary graph into two smaller induced subgraphs we have to decompose a graph by partitioning vertices hence we define a procedure for decomposing a randomly connected graph in this manner.

RandomPartitionInduced (Algorithm\ref{alg:alg09}) decomposes a connected graph into two connected induced subgraphs.
First, the algorithm initializes two induced subgraphs by distributed vertices and the edges that the vertices induce.
Then, two subgraphs transfer their minimum components until they become connected, a process similar to the one in the algorithm for subgraph query.

The decomposition of a disconnected graph is identical to the one for subgraph query.

\begin{algorithm}
\caption{RandomPartitionInduced}
\label{alg:alg09}
\begin{algorithmic}
\STATE Input: 5-Tuple of Graph $(g,s,P,\{c^{'},c^{''}\},E)$
\STATE Output: 5-Tuple of a pair of graphs $(c^{'},s,P_1 ,\{c^{'}_1 ,c^{''}_1 \},E_1 )$, $(c^{''},s,P_2 ,\{c^{'}_2 ,c^{''}_2 \},E_2 )$
\end{algorithmic}
\begin{algorithmic}[1]
\STATE  $c^{'} \leftarrow  \emptyset$
\STATE  $c^{''} \leftarrow \emptyset$
\STATE  $s_1 \leftarrow s$
\STATE  $s_2 \leftarrow s$
%%\STATE  $\{V_{c^{'}} , V_{c^{''}}\} \leftarrow V_g $ 
%%\STATE  partition at random $V_g $ to $V_{c^{'}}$ and $V_{c^{''}}$
%%\STATE $| V_{c^{'}} | = | V_g / 2 |$
%%\STATE $| V_{c^{''}} | = | V_g | - | V_{c^{'}} |$
\WHILE {$|V_{c^{'}}| \leq   \lfloor |V_g |/2 \rfloor $} \DO
       \STATE $v \leftarrow$ random vertex of $ V_g $
       \STATE $V_{c^{'}} \leftarrow V_{c^{'}} \cup v $ 
       \STATE  $ V_g \leftarrow V_g - v $
\ENDWHILE
\STATE $ V_{c^{''}} \leftarrow V_g $
\STATE  $E_{c^{'}} \leftarrow $ Edges induced in $V_{c^{'}}$ by $g $
\STATE  $E_{c^{''}} \leftarrow $ Edges induced in $V_{c^{''}}$ by $g$
\STATE  $c^{'} \leftarrow \{V_{c^{'}} , E_{c^{'}}  \}$ 
\STATE  $c^{''} \leftarrow \{V_{c^{''}} , E_{c^{''}}  \}$ 
%%\STATE $E_1 \leftarrow  E \cap ( E_{c^{'}} \cup E_{c^{''}} )$
%%\STATE $E_2 \leftarrow  E \cap ( E_{c^{'}} \cup E_{c^{''}} )$
%%\STATE $P_1 \leftarrow \{g\}$
%%\STATE $P_2 \leftarrow \{g\}$
\REPEAT
%%\STATE $E_{c^{'}} \leftarrow \emptyset $
%%\STATE $E_{c^{''}} \leftarrow \emptyset $
	\IF{ $c^{'}$ \AND $c^{''}$ are disconnected}
             \IF{ $|c^{'}| > |c^{''}|$ }
                  \STATE $  c^{''} \leftarrow c^{''} \cup  c^{'}_{min} $
             \ELSE
                \STATE $  c^{'} \leftarrow c^{'} \cup c^{''}_{min} $
%%              \STATE $min \{c^{'} , c^{''} \} \leftarrow min \{c^{'} , c^{''} \} \cup $ minimal component of $max\{c^{'} , c^{''} \}$ 
%%		\STATE select larger graph from $\{c^{'} , c^{''} \}$ 
%%              \STATE move minimal component to opposite graph
             \ENDIF
        \ELSE
                \IF { $c^{'}$ is disconnected}
                   \STATE $  c^{''} \leftarrow c^{''} \cup c^{'}_{min} $
%                  \STATE select disconnected graph from $\{c^{'} , c^{''} \}$
%                  \STATE move minimal component to opposite graph
                \ELSE
                   \STATE $  c^{'} \leftarrow c^{'} \cup c^{''}_{min}$ 
                \ENDIF
        \ENDIF
%%	\STATE reset edges induced by distributed vertices to $c^{'}$ and $c^{''}$ 

\UNTIL{ $c^{'}$ \AND $c^{''}$ are connected}
\STATE $E_1 \leftarrow  E -  ( E_{c^{'}} \cup E_{c^{''}} )$
\STATE $E_2 \leftarrow  E -  ( E_{c^{'}} \cup E_{c^{''}} )$
\STATE $P_1 \leftarrow \{g\}$
\STATE $P_2 \leftarrow \{g\}$
\RETURN  ($(c^{'},s,P,\{c^{'}_1 ,c^{''}_1 \},E_1 )$, $(c^{''},s,P,\{c^{'}_2 ,c^{''}_2 \},E_2 )$)
\end{algorithmic}
\end{algorithm}

\subsubsection{Query Processing}
The search processing algorithms for subgraphs, \textit{NNA} (Algorithm\ref{alg:alg05}), SubgraphQuery (Algorithm\ref{alg:aalg06}) and AssignVertex(Algorithm\ref{alg:alg07}) are also applicable to induced subgraph query. However, due to the difference in the method of decomposition, we need a new procedure for the combining the 
induced subgraph isomorphisms. CombineInduced(Algorithm\ref{alg:alg010}) outlines that procedure.
When CombineInduced is called with graph $g$ as input, the sets of induced subgraph isomorphisms $F_1, F_2$ from its children $c^{'}, c^{''}$ 
to the query graph must have been already calculated, as part recursive calling of \label{alg:aalg06}.
There are two conditions that must be satisfied to CombineInduced two functions $f_1 \in F_1 and f_2 \in F_2$.
First, there must exist edges in the query graph correspond to cut edges and vice versa(line 7).
Second, we must ensure that $f_1$ and $f_2$ are disjoint(line 8), otherwise the combination of $f_1$ and $f_2$ will not be injective function.
CombineInduced algorithm checks all pairs of the two functions and returns the sets of its unions that satisfy these conditions.

\begin{algorithm}[t]
\caption{CombineInduced}
\label{alg:alg010}
\begin{algorithmic}
\STATE Input:  Graph 5-Tuple $(g,s,P,\{c^{'},c^{''}\},E)$, Query graph $q$
\STATE Output: Subgraph isomorphisms $F$
\end{algorithmic}
\begin{algorithmic}[1]
%%\STATE let $c^{'}$ and $c^{''}$ be children of $g$
%%\STATE let $\bar{p_1}$ and $\bar{p_2}$ be functions from $g$ to $c^{'}$ and $c^{''}$
%%\STATE let $F_1$ and $F_2$ be sets of injection from $c^{'}$ and $c^{''}$ to query graph
\STATE $F_1 \leftarrow$CombineInduced$(c^{'},s_1 ,P_1 ,\{c_1^{'},c_1^{''}\},E_1 ,q)$
\STATE $F_2 \leftarrow$CombineInduced$(c^{''},s_2 ,P_2 ,\{c_2^{'},c_2^{''}\},E_2 ,q)$
\STATE $F \leftarrow \emptyset$
\STATE $E \leftarrow E_g - (E_1 \cup E_2 )$
\FOR{each pair $(f_1, f_2 )$ in $(F_1, F_2 )$}
		\IF{ for each $e = (v_1, v_2) \in E$ there exists  $e_q = (f_1(v_1),f_2(v_2)) \in E_q$ with $\nu(e) = \nu_q(e_q)$ 
\STATE \AND for each $e_q = (u_1, u_2) \in E_q$ such that $u_1 \in f_1(V_{c^{'}}), u_2 \in  f_2(V_{c^{''}}) $ there exists $e = (f_1^{-1}(u_1), f_2^{-1}(u_2)) \in E$ with $\nu_q(e_q) = \nu(e)$}	
			\IF{$f_1(V_{c^{'}}) \cap f_2(V_{c^{''}}) = \emptyset$}
                                \STATE $g \leftarrow (c^{'} \cup_{E} c^{''})$
                                \STATE $V_{g} \leftarrow f(V_{c^{'}} \cup V_{c^{''}} })$
                                \STATE $F \leftarrow  F \cup \{ f \} $
%%				\STATE define subgraph isomorphism $f$ by eq(\ref{eq:eq1})
%%				\STATE $F = F \cup f $
		\ENDIF
	\ENDIF
\ENDFOR
\RETURN $F$
\end{algorithmic}
\end{algorithm}

\subsection{Network Algorithm with Graph Pruning}
The \textit{Network Algorithm} cannot process a local query efficiently (i.e. In the case that you want to know subgraph isomorphisms for only some graphs in a graph database to a query graph, the \textit{Network Algorithm} calculates subgraph isomorphisms for all the nodes in $DAG$).
In contrast our recursive formulation of the \textit{Network Algorithm} only checks the relevant descendants of selected nodes and this allows us to solve local queries more efficiently.

One result of the support of local query is that we can make use of some graph pruning techniques. Let us consider a simple example \textit{label check}. 
If a graph database stores the numeric labels of attributed graphs, it may be easily determined when they have no subgraph isomorphisms to a query graph just by comparing the attributes of the graph with those of the query graph.
In this way, we can efficiently reduce the candidates graphs for the answer.
Consequently, by pruning graphs in advance, we can significantly reduce the related procesing time.


% LocalWords:  EdgeIDs
