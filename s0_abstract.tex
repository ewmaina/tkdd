
Graphs have become increasingly important structures for representing and understanding complex structures.However enumerating all subgraph isomorphisms is hard because it requires repeated solution of the subgraph isomorphism problem which is NP Complete.
We propose a new method for scalably and efficiently 
enumerating all subgraph isomorphisms of a query graph that exist in a graph database. The new approach is based on a highly compact graph database and an efficient search strategy for graphs in the graph database.

We start by recursively decomposing each database graph into connected subgraphs and then construct a graph database by storing all its subgraphs at the nodes of a rooted Directed Acyclic Graph (DAG). To process a query graph, we use the DAG to apply three criteria to reduce unnecessary computation of subgraph isomorphism. First, if all the vertices in the database graph do not exist in the query then that database graph is not a subgraph of the query graph. Secondly, if any subgraph of the database graph does not exist in the query graph, the database graph is not a subgraph of the query graph. Thirdly, if a common subgraph of some database graphs does not exist in the query graph, none of the database graphs is a subgraph of the query graph. The criteria are implemented as a traversal of the DAG data structure starting at the root and searching each database graph. Application of the conditions results in compact search space. The search is further optimized by flagging out non-isomorphic graphs as soon as one of the above rules is broken. 

We present experimental results where we compare our proposed algorithms with two well known algorithms, an all isomorphism enumeration algorithm by Messmer et. al and a state of the art isomorphism detection only algorithm, VF2. Results show an order of magnitude improvement in scalability over Messmer et. al. as well as substantial improvement over the VF2 algorithm despite the increased workload of enumerating all subgraph isomorphisms. Our method is particularly suited to larger query graphs or very larger graph databases
